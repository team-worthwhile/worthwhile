\section{Beweiserschnittstelle}%

Die Beweiserschnittstelle lässt Worthwhile-""Programmspezifikationen
und -Annotationen mit Voraussetzungen von einem Beweiser überprüfen
und liefert dem Aufrufer dessen Auswertung zurück.%

\subsection{Eingabedaten}%

\subsubsection{Programmspezifikation}%

% TODO Datentyp des Abstract-Syntax-Tree

Die Programmspezifikation wird dem Abstract-Syntax-Tree~(AST), welcher
aus einem Worthwhile-Text erstellt wurde, entnommen. Sie setzt sich
zusammen aus dem WHILE-Programmtext und den Annotationen. Annotationen
sind entweder Axiome, Funktionsverträge, Schleifeninvarianten,
Zusicherungen oder Voraussetzungen.%

\subsubsection{Annotationen mit Voraussetzungen}%

Der übergebene AST besteht aus genau einer Annotation, insbesondere
fehlt jeglicher Programmtext. Allerdings wird zusätzlich eine Menge
von Voraussetzungen übergeben, welche die Beweiserschnittstelle nicht
selbst dem AST entnehmen kann.%

% TODO was geschieht, wenn sich die Voraussetzungen widersprechen
% TODO Beispiel finden, für das ausgewertete Zusicherungen
%      übergeben werden müssen

Diese Voraussetzungen enthalten den Zustand einer Programmausführung
und bereits ausgewertete Annotationen. Der Programmzustand besteht aus
den von einem Interpreter berechneten Wertbelegungen von Symbolen, die
insbesondere in der auszuwertenden Annotation vorkommen können. Den
Programmzustand kann die Beweiserschnittstelle nicht selbst
feststellen, also auch nicht mehr zu einem früheren zurückkehren oder
zu einem späteren übergehen und andere Annotationen auswerten. Deshalb
werden in den Voraussetzungen auch die zuvor zu \texttt{true}
ausgewertete Annotationen enthalten sein.%

\subsection{Formelerstellung}%

Bei der Erstellung einer Formel aus einer Programmspezifikation wird
nach dem Kalkül der schwächsten Vorbedingung vorgegangen.%

\subsubsection{Axiome}%

Die schließlich an den Beweiser übergebene Formel ist eine
Implikation, auf deren linke Seite die konjunktiv verknüpften Axiome
stehen. Wenn keine Axiome angegeben wurden, steht auf der linke Seite
der Implikation der Wahrheitswert \texttt{true}. Auf der rechten Seite
steht das zu beweisende Theorem, dass das Programm der Spezifikation
entspricht.%

Axiome sind im AST als \texttt{axiom}-Knoten codiert und die Kinder
sind syntaktische Elemente prädikatenlogische Formeln.%

\subsubsection{Funktionsverträge}%

Funktionen werden modular spezifiziert. Ihre Spezifikation besteht aus
den Voraussetzungen, die ihre Parameter erfüllen müssen, und den
Zusicherungen, die sie bei gültigen Parametern für ihr Ergebnis
erfüllen. Voraussetzungen werden Vorbedingungen genannt, Zusicherungen
Nachbedingungen und die Menge aller Bedingungen Funktionsvertrag.%

Für den Funktionstext werden die Vorbedingungen wie Axiome behandelt
und Nachbedingungen wie Zusicherungen für den Rückgabewert der
Funktion.%

Vorbedingungen werden im AST als \texttt{requires}-Knoten und
Nachbedingungen als \texttt{ensures}-Knoten codiert. Die Kinder beider
Knotentypen sind syntaktische Elemente prädikatenlogischer Formeln.%

\subsubsection{Schleifeninvarianten}%

Schleifeninvarianten werden in der Formel an den Beweiser so
eingebettet, dass ihre Aussage vor dem Schleifeneintritt, nach jedem
Durchlauf und nach dem Austritt gelten muss. Nach jedem Durchlauf
heißt, dass die Erfülltheit der Schleifenbedingung und der Invariante
die Aussage der Invariante impliziert, und nach dem Austritt bedeutet
analog, dass die Erfülltheit der negierten Schleifenbedingung und der
Invariante wiederum die Aussage der Invariante impliziert.%

Schleifeninvarianten werden im AST als \texttt{invariant}-Knoten
codiert, deren Kinder syntaktische Elemente prädikatenlogischer
Formeln sind.%

\subsubsection{Zusicherungen}%

Zusicherungen werden in der Formel an den Beweiser so eingebettet,
dass ihre Aussage bei jeder Programmausführung durch den
vorhergehenden Programmtext erfüllt sein muss.%

Zusicherungen werden im AST als \texttt{assert}-Knoten codiert, deren
Kinder syntaktische Elemente prädikatenlogischer Formeln sind.%

\subsubsection{Annahmen}%

Annahmen werden wie Axiome behandelt, sodass sie insbesondere nicht
wie Zusicherungen erfüllt sein müssen und vom Beweiser auch nicht
geprüft werden.%

Voraussetzungen werden im AST als \texttt{assume}-Knoten codiert,
deren Kinder syntaktische Elemente prädikatenlogischer Formeln sind.%

\subsection{Voraussetzungen für einzelne Annotationen}%

% TODO Datentyp für Programmzustand und ausgewertete Annotationen

Voraussetzungen bei der Prüfung von einzelnen Annotationen werden wie
Axiome behandelt.%

\subsection{Ausgabedaten}%

\subsubsection{Modell}%

\subsection{Klassenentwurf}%

\begin{itemize}%

    \item Zur Transformation einer Programmspezifikation werden für
    \texttt{WWAST} Besucherklassen für die einzelnen syntaktischen
    Elemente implementiert.%

    % TODO Spezifikation des unterstützten SMT-LIB-Standards

\end{itemize}%

\subsubsection{class Environment}%

Instanzen von \texttt{Environment} fassen einen Programmzustand
zusammen. Dazu gehören sowohl Wertbelegungen für Symbole als auch
erfüllte Annotationen.%

\subsubsection{class WWAST}%

Instanzen von \texttt{WWAST} codieren Programmtexte, welche der Syntax
von Worthwhile entsprechen.%

\subsubsection{class SMTLIBAST}%

Instanzen von \texttt{SMTLIBAST} codieren Programmtexte, welche der
Syntax von SMT-LIB entsprechen.%

\subsubsection{class ProverInterface}%

Instanzen von \texttt{ProverInterface} übersetzen
\texttt{WWAST}-Instanzen in Instanzen von \texttt{SMTLIBAST}.%

% TODO nicht öffentliche Attribute

\begin{description}%
    \item [private int timeout]

    Zeit in Sekunden, nach der \texttt{check""Program} und
    \texttt{check""Annotation} zurückkehren müssen.%

    Siehe \texttt{setTimeout}.%

    \item [private SMTLIBAST model]%

\end{description}%

% TODO öffentliche Methoden

\begin{description}%
    \item[public bool checkProgram(WWAST prg)]

    Prüft mit Hilfe eines Beweisers die Korrektheit des annotierten
    Programms \texttt{prg}. Es wird genau dann \texttt{true}
    zurückgeliefert, wenn \texttt{prg} für alle Ausführungen den
    Annotationen entspricht, und \texttt{false} sonst~(z.~B.\, wenn
    \texttt{timeout} überschritten wurde oder die Spezifikation für
    eine Ausführung nicht erfüllt wird).%

    Siehe \texttt{getModel}%

    \item[public bool checkAnnotation(WWAST ann, Environment env)]

    Prüft mit Hilfe eines Beweisers die Erfülltheit der Annotation
    \texttt{ann}. Es wird genau dann \texttt{true} zurückgeliefert,
    wenn \texttt{ann} im Kontext \texttt{env} erfüllt ist, und
    \texttt{false} sonst~(z.~B.\, wenn \texttt{timeout} überschritten
    wurde).%

    Siehe \texttt{getModel}%

    \item[public Environment getModel()]

    Liefert für den vorhergehenden Aufruf von \texttt{check""Program}
    oder \texttt{check""Annotation} die Ausgabe des Beweisers und
    \texttt{null}, wenn zuvor weder ein Programm noch eine Annotation
    überprüft wurde oder keine Ausgabe verfügbar war.%

    \item[public void setTimeout(int seconds)]

    Setzt den Timeout für Aufrufe von \texttt{check""Program} und
    \texttt{check""Annotation} auf den Wert von \texttt{seconds}. Ist
    der angegebene Wert negativ, wird er als Null interpretiert.%

    \item[public SMTLIBAST transformProgram(WWAST prg)]

    Liefert eine SMT-""LIB-""Formel für das annotierte Programm
    \texttt{prg}, wie sie von \texttt{check""Program} als Eingabe für
    den Beweiser erstellt würde.%

    \item[public SMTLIBAST transformAnnotation(WWAST ann, Environment env)]

    Liefert eine SMT-""LIB-""Formel für die Annotation \texttt{ann} im
    Kontext \texttt{env}, wie sie von \texttt{check""Annotation} als
    Eingabe für den Beweiser erstellt würde.%

\end{description}%

% TODO nicht öffentliche Methoden

\begin{description}%
    \item[private SMTLIBAST transformWhile(WWAST loop)]%

    \item[private SMTLIBAST transformFunction(WWAST fun)]%

    \item[private SMTLIBAST transformFunCall(WWAST call)]%

    \item[private SMTLIBAST transformAssign(WWAST assign)]%

    \item[private SMTLIBAST transfromConditional(WWAST if)]%

    \item[private SMTLIBAST transfromAssert(WWAST assert)]%

    \item[private SMTLIBAST transfromAssume(WWAST assume)]%

    \item[private SMTLIBAST transformAxiom(WWAST axiom)]%

    \item[private bool prove(SMTLIBAST formula)]

    Aufruf des Beweisers für die SMT-""LIB-""Formel \texttt{formula}.
    Es wird genau dann \texttt{true} zurückgeliefert, wenn die Formel
    erfüllt werden konnte, und \texttt{false} sonst~(z.~B.\, wenn
    \texttt{timeout} überschritten wurde). Hat im Falle der
    Erfüllbarkeit der Beweiser ein Modell geliefert, wird dies der
    Variable \texttt{model} entsprechend zugewiesen.%

\end{description}%

\subsection{Interaktionsentwurf}%
