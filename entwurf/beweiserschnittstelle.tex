\section{Beweiserschnittstelle}%

Die Beweiserschnittstelle lässt Worthwhile-""Programmspezifikationen
und -Annotationen mit Voraussetzungen von einem Beweiser überprüfen
und liefert dem Aufrufer dessen Auswertung zurück.%

\subsection{Eingabedaten}%

\subsubsection{Programmspezifikation}%

% TODO Datentyp des Abstract-Syntax-Tree

Die Programmspezifikation wird dem Abstract-Syntax-Tree~(AST), welcher
aus einem Worthwhile-Text erstellt wurde, entnommen. Sie setzt sich
zusammen aus dem WHILE-Programmtext und den Annotationen. Annotationen
sind entweder Axiome, Funktionsverträge, Schleifeninvarianten,
Zusicherungen oder Annahmen.%

\subsubsection{Annotationen mit Voraussetzungen}%

Der übergebene AST besteht aus genau einer Annotation, insbesondere
fehlt jeglicher Programmtext. Allerdings wird zusätzlich eine Menge
von Voraussetzungen übergeben, welche die Beweiserschnittstelle nicht
selbst dem AST entnehmen kann.%

% TODO was geschieht, wenn sich die Voraussetzungen widersprechen
% TODO Beispiel finden, für das ausgewertete Zusicherungen
%      übergeben werden müssen

Diese Voraussetzungen enthalten den Zustand einer Programmausführung.
Der Programmzustand enthält
die von einem Interpreter berechneten Wertbelegungen von Symbolen, die
insbesondere in der auszuwertenden Annotation vorkommen können. Den
Programmzustand kann die Beweiserschnittstelle nicht selbst
feststellen.

\subsection{Formelerstellung im wp-Kalkül}%

\subsubsection{Axiome}%

Die schließlich an den Beweiser übergebene Formel ist eine
Implikation, auf deren linke Seite die konjunktiv verknüpften Axiome
stehen. Wenn keine Axiome angegeben wurden, steht auf der linke Seite
der Implikation der Wahrheitswert \texttt{true}. Auf der rechten Seite
steht das zu beweisende Theorem, dass das Programm der Spezifikation
entspricht.%

Axiome sind im AST als \texttt{axiom}-Knoten codiert und die Kinder
sind syntaktische Elemente prädikatenlogische Formeln.%

\subsubsection{Funktionsverträge}%

Funktionen werden modular spezifiziert. Ihre Spezifikation besteht aus
den Annahmen, die ihre Parameter erfüllen müssen, und den
Zusicherungen, die sie bei gültigen Parametern für ihr Ergebnis
erfüllen. Annahmen werden Vorbedingungen genannt, Zusicherungen
Nachbedingungen und die Menge aller Bedingungen Funktionsvertrag.%

Für den Funktionstext werden die Vorbedingungen wie Axiome behandelt
und Nachbedingungen wie Zusicherungen für den Rückgabewert der
Funktion.%

Vorbedingungen werden im AST als \texttt{requires}-Knoten und
Nachbedingungen als \texttt{ensures}-Knoten codiert. Die Kinder beider
Knotentypen sind syntaktische Elemente prädikatenlogischer Formeln.%

Siehe \texttt{FP020}%

\subsubsection{Schleifeninvarianten}%

Schleifeninvarianten werden in der Formel an den Beweiser so
eingebettet, dass ihre Aussage vor dem Schleifeneintritt, nach jedem
Durchlauf und nach dem Austritt gelten muss. Nach jedem Durchlauf
heißt, dass die Erfülltheit der Schleifenbedingung und der Invariante
die Aussage der Invariante impliziert, und nach dem Austritt bedeutet
analog, dass die Erfülltheit der negierten Schleifenbedingung und der
Invariante wiederum die Aussage der Invariante impliziert.%

Schleifeninvarianten werden im AST als \texttt{invariant}-Knoten
codiert, deren Kinder syntaktische Elemente prädikatenlogischer
Formeln sind.%

Siehe \texttt{FP030}%

\subsubsection{Zusicherungen}%

Zusicherungen werden in der Formel an den Beweiser so eingebettet,
dass ihre Aussage bei jeder Programmausführung durch den
vorhergehenden Programmtext erfüllt sein muss.%

Zusicherungen werden im AST als \texttt{assert}-Knoten codiert, deren
Kinder syntaktische Elemente prädikatenlogischer Formeln sind.%

\subsubsection{Annahmen}%

Annahmen werden wie Axiome behandelt, sodass sie insbesondere nicht
wie Zusicherungen erfüllt sein müssen und vom Beweiser auch nicht
geprüft werden.%

Annahmen werden im AST als \texttt{assume}-Knoten codiert,
deren Kinder syntaktische Elemente prädikatenlogischer Formeln sind.%

\subsubsection{Arrayzugriffe}%

Für Arrays werden implizit die drei Axiome (A1), (A2) und (A3)
vorausgesetzt. Außerdem wird jedem Arrayzugriff im Programmtext bei
der Formelerstellung die Zusicherung, dass der angegebene Feldindex
nicht die Deklarationsgröße übersteigt, vorangestellt.%

\begin{description}%
    \item[A1] \begin{math}\forall i \in \mathbb{A}, e \in \mathbb{B} : \{\texttt{true}\} a[i] := e \{a[i] = e\}\end{math}%
    \item[A2] \begin{math}\forall i, j \in \mathbb{A}, e, f \in \mathbb{B} : \{i \neq j \wedge a[i] = e\} a[i] := f \{a[i] = e\}\end{math}%
    \item[A3] \begin{math}\forall a, b \in \mathbb{B}^\mathbb{A} : (\forall i, j \in \mathbb{A} : a[i] = b[j]) \Rightarrow a = b\end{math}%
\end{description}%

\subsubsection{Division}%

Ausdrücken, die eine Division durchführen, wird implizit die Zusicherung
vorangestellt, dass der Dividend ungleich Null ist. Damit wird erreicht, dass
die Möglichkeit eines Programmdurchlaufs, der eine Division durch Null enthält,
auch bei der Prüfung der Spezifikation erkannt und das spezifizierte Programm
nicht als korrekt eingestuft wird, obwohl nicht alle Programmaufrufe mit
definiertem Ergebnis terminieren.%

\subsubsection{Typsystem}%

Die in Worthwhile vorhandenen Datentypen \texttt{Integer} und
\texttt{Boolean} werden beide von dem Beweiser Z3 unterstützt. Typen
werden bei Parameterübergabe, Symbolreferenz und Symboldefinition
geprüft. Treten nicht übereinstimmende Typen auf, liefert Z3 einen
Fehler.%

\subsection{Voraussetzungen für einzelne Annotationen}%

% TODO Datentyp für Programmzustand und ausgewertete Annotationen

Voraussetzungen bei der Prüfung von einzelnen Annotationen werden wie
Axiome behandelt.%

\subsection{Ausgabedaten}%

\subsubsection{Modell}%

\subsection{Klassenentwurf}%

\begin{itemize}%

    \item Zur Transformation einer Programmspezifikation werden für
    \texttt{WWAST} Besucherklassen für die einzelnen syntaktischen
    Elemente implementiert.%

    \item Die Besucher hängen von Strategieklassen ab, welche die
    Transformationsregeln und die Formelsprache austauschbar machen.%

    % TODO Spezifikation des unterstützten SMT-LIB-Standards

\end{itemize}%

\subsubsection{class Environment}%

Instanzen von \texttt{Environment} fassen einen Programmzustand
zusammen. Dazu gehören sowohl Wertbelegungen für Symbole als auch
erfüllte Annotationen.%

\subsubsection{class WWAST}%

Instanzen von \texttt{WWAST} codieren Programmtexte, welche der Syntax
von Worthwhile entsprechen.%

\subsubsection{class SMTLIBAST}%

Instanzen von \texttt{SMTLIBAST} codieren Programmtexte, welche der
Syntax von SMT-LIB entsprechen.%

\subsubsection{class Prover::Formula}%

Formula beschreibt Instanzen, die prädikatenlogische Formeln als
Syntax~Tree \footnote{Die Syntax prädikatenlogischer Formeln ist
\url{http://formal.iti.kit.edu/teaching/pse/2011/Downloads/intro.pdf}
entnommen} zugänglich machen.%

\subsubsection{class Prover::SpecificationChecker}%

Instanzen von \texttt{SpecificationChecker} übersetzen
\texttt{WWAST}-Instanzen in Instanzen von \texttt{SMTLIBAST}.%

% TODO nicht öffentliche Attribute

\begin{description}%
    \item [private int timeout]

    Zeit in Sekunden, nach der \texttt{check""Program} und
    \texttt{check""Annotation} zurückkehren müssen.%

    Siehe \texttt{setTimeout}.%

    \item [private SMTLIBAST model]%

\end{description}%

% TODO öffentliche Methoden

\begin{description}%
    \item[public SpecificationChecker(ProverCaller prover, FormulaGenerator generator)]%

    \item[public bool checkProgram(WWAST program)]

    Prüft mit Hilfe eines Beweisers die Korrektheit des annotierten
    Programms \texttt{prg}. Es wird genau dann \texttt{true}
    zurückgeliefert, wenn \texttt{prg} für alle Ausführungen den
    Annotationen entspricht, und \texttt{false} sonst~(z.~B.\, wenn
    \texttt{timeout} überschritten wurde oder die Spezifikation für
    eine Ausführung nicht erfüllt wird).%

    Siehe \texttt{getModel}%

    Siehe \texttt{FP005}%

    \item[public bool checkAnnotation(WWAST annotation, Environment env)]

    Prüft mit Hilfe eines Beweisers die Erfülltheit der Annotation
    \texttt{ann}. Es wird genau dann \texttt{true} zurückgeliefert,
    wenn \texttt{ann} im Kontext \texttt{env} erfüllt ist, und
    \texttt{false} sonst~(z.~B.\, wenn \texttt{timeout} überschritten
    wurde).%

    Siehe \texttt{getModel}%

    Siehe \texttt{FP005}%

    \item[public Environment getModel()]

    Liefert für den vorhergehenden Aufruf von \texttt{check""Program}
    oder \texttt{check""Annotation} die Ausgabe des Beweisers und
    \texttt{null}, wenn zuvor weder ein Programm noch eine Annotation
    überprüft wurde oder keine Ausgabe verfügbar war.%

    \item[public void setTimeout(int seconds)]

    Setzt den Timeout für Aufrufe von \texttt{check""Program} und
    \texttt{check""Annotation} auf den Wert von \texttt{seconds}. Ist
    der angegebene Wert negativ, wird er als Null interpretiert.%

\end{description}%

\subsubsection{interface Prover::ProgramTransformer::FormulaGenerator}%

\begin{description}%
    \item[public Formula transformLoop(WWAST statement)]%

    \item[public Formula transformFunctionDefinition(WWAST statement)]%

    \item[public Formula transformEnsures(WWAST statement)]%

    \item[public Formula transformRequires(WWAST statement)]%

    \item[public Formula transformInvariant(WWAST statement)]%

    \item[public Formula transformFunctionCall(WWAST statement)]%

    \item[public Formula transformVariableAssignment(WWAST statement)]%

    \item[public Formula transformArrayAssignment(WWAST statement)]%

    \item[public Formula transfromConditional(WWAST statement)]%

    \item[public Formula transformArrayElementAssignment(WWAST statement)]%

    \item[public Formula transfromAssert(WWAST statement)]%

    \item[public Formula transfromAssume(WWAST statement)]%

    \item[public Formula transformAxiom(WWAST statement)]%

    \item[public Formula transformVariableDeclaration(WWAST statement)]%

\end{description}%

\subsubsection{class Prover::TheoremProver::ProverCaller}%

\begin{description}%
    \item[public ProverResult checkFormula(Formula formula)]

    Aufruf des Beweisers für die prädikatenlogische Formel
    \texttt{formula}. Es wird genau dann \texttt{true} in
    \texttt{ProverResult\#satisfiable} zurückgeliefert, wenn die Formel
    erfüllt werden konnte, und \texttt{false} sonst. Hat im Falle der
    Erfüllbarkeit der Beweiser ein Modell geliefert, wird dies in
    \texttt{ProverResult\#model} entsprechend zurückgeliefert.%

\end{description}%

\subsubsection{class Prover::TheoremProver::StdProver extends ProverCaller}%

\begin{description}%
    \item[public StdProver(String path, FormulaCompiler compiler)]%

\end{description}%

\subsubsection{interface Prover::TheoremProver::FormulaCompiler}%

\begin{description}%
    % TODO return types missing
    \item[public compileForall(Formula formula)]%

    \item[public compileExists(Formula formula)]%

    \item[public compileConjunction(Formula formula)]%

    \item[public compileDisjunction(Formula formula)]%

    \item[public compileNegation(Formula formula)]%

    \item[public compileImplication(Formula formula)]%

\end{description}%

\subsection{Interaktionsentwurf}%
