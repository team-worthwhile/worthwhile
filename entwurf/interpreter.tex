\section{Interpreter}

Der Interpreter führt in der WHILE-""Sprache formulierte Programme aus und stellt eine Schnittstelle zum Debuggen von Programmen zur Laufzeit bereit. Des weiteren überprüft er mit Hilfe des Evaluators zur Laufzeit die mittels Annotationen in den Programmtext eingebetteten Annotationen. Der Interpreter läuft in einem eigenen Thread und interagiert mit Hilfe von gemeinsam genutzten Datenstrukturen mit dem GUI-Thread.

\subsection{Eingabedaten}

\subsubsection{Programmspezifikation}

% copy-pasta von Beweiserschnittstelle oder ein neuer Abschnitt, der diesen Begriff einmal definiert?

\subsubsection{Klassenentwurf}

\subsubsection{class ExecutionContext}
Instanzen von \texttt{ExecutionContext} verwalten den Zustand eines Ausführungskontext. Dazu gehören ein \texttt{ExpressionEvaluator}, ein \texttt{StatementExecutor} sowie eine Symboltabelle, in der in diesem Kontext angelegte Symbole gespeichert werden.

\textbf{Öffentliche Methoden}
\begin{description}
    \item[public SymbolTable getSymbolTable()]
    Gibt die aktuelle Symboltabelle zurück.

    \item[public void executeStatement(Statement statement)]
    Führt in diesem Kontext das angegebene Statement aus.

    \item[public void evaluateExpression(Statement statement)]
    Wertet in diesem Kontext die gegebene Expression aus und gibt das Ergebnis der Auswertung zurück.
\end{description}

\textbf{Private Methoden und Attribute}
\begin{description}
    \item[private SymbolTable symbolTable]
    \item[private SymbolTable statementExecutor]
    \item[private SymbolTable expressionEvaluator]
\end{description}


\subsubsection{class Interpreter}

Instanzen von \texttt{Interpreter} kontrollieren die Ausführung eines initial übergebenen Programms.

\begin{description}
    \item[public Interpreter(Program p, DebugInterface debugInterface)]
    Erzeugt ein Instanz, das das Programm mit der gegebenen Spezifikation \texttt{p} ausführen kann.

    \item[public void executeStatement(Statement s)]
    Führt das gegebene Statement aus und aktualisiert bei Bedarf die Symboltabelle.
\end{description}


\subsubsection{interface ExpressionEvaluator}
\subsubsection{interface DebugInterface}
Implementationen \texttt{DebugInterface} sind für die Interaktion des Interpreters mit anderen Teilen des Programms zuständig.

\begin{description}
    \item[public void statementExecuted()]
    Wird vom Interpreter nach jedem Ausführen eines Statements aufgerufen.
    \item[public void expressionEvaluated()]
    Wird vom Interpreter nach jedem Auswerten jeder Expression aufgerufen.
    \item[public void executionStarted()]
    Wird vom Interpreter direkt nach dessen Start aufgerufen.
    \item[public void executionTerminatedNormally()]
    Wird vom Interpreter nach ordnungsgemäßer Terminierung des Programms aufgerufen.
    \item[public void executionTerminatedAbnormally()]
    Wird vom Interpreter nach einer fehlerbedingten Terminierung des Programms, ausgelöst beispielsweise durch einen Laufzeitfehler, aufgerufen.
\end{description}
