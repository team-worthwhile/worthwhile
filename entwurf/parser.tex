\section{Parser}

Der Parser bietet die Funktionaltiät einen Abstract-""Syntax-""Trees aus einem Programm in while Sprache zu generieren und dabei auftretende Syntax Fehler zu übergeben.

\subsection{Klassenentwurf}

\subsubsection{class Parser}

Die Klasse Parser dient als Schnittstelle zum externen Parser und bündelt komplexe Vorgänge in einzelnen Methodenaufrufen. Hier kommt das Entwurfsmuster "`Fasade"' zur Anwendung.

\begin{description}
	\method{public Parser(source : String)}
	\method{public Parser(source : Readable)}
		Der Konstruktor der Klasse kann wahlweise mit einem Parameter, welcher das zu parsende Programm in String Form enthält, oder mittels eines "`Readable"' Objekts, aufgerufen werden.

	\method{public INode[*] getSyntaxErrors()}
		Diese Methode liefert eine Liste der Syntax Fehler, die beim parsen erkannt wurden, zurück.
  
	\method{public ASTNode getRootASTElement()}
		Hiermit wird das Wurzelelement des Abstract-""Syntax-""Trees, des geparsten Programms, zurück gegeben.

	\method{public Boolean hasSyntaxErrors()}
		Der Rückgabewert dieser Methode gibt an, ob während des parsens Syntax Fehler aufgetreten sind.
\end{description}

\subsubsection{class Validator}

Diese Klasse wird komponentenintern zur semantischen Validierung des geparsten Programms benutzt. Hierzu wird ein programmunabhängiges Typsystem benutzt.

\begin{description}
	\method{public Validator()}
		Diese Methode initialisiert eine neue Instanz der Klasse Validator.

	\method{public checkTypesystemRules(node : ASTNode)}
		Überprüft den Abstract-""Syntax-""Tree des übergebenen ASTNodes auf semantische Fehler
\end{description}

\subsubsection{class TypeSystemCheck}

Diese Klasse ist ein Teil des programmunabhängiges Typsystem.

\begin{description}
	\method{public checkTypesystemConstraints(node : ASTNode, validator : Validator)}
		Dies ist die einzige Methode, die vom Validator zum überprüfen aufgerufen wird.
\end{description}
