\documentclass[t]{beamer}
\usetheme[deutsch]{KIT}
\setbeamercovered{transparent}
\setbeamertemplate{navigation symbols}{}

\KITfoot{Worthwhile - Praxis der Softwareentwicklung WS 2011/2012}
\usepackage[utf8]{inputenc}
\usepackage{ngerman}
\usenavigationsymbols

\title{Worthwhile}
\subtitle{Abschlusspräsentation Entwurfsphase}
\author{Leon Handreke $\cdot$ Chris Hiatt $\cdot$ Stefan Orf $\cdot$ Joachim Priesner $\cdot$ Fabian Ruch $\cdot$ Matthias Wagner}

\institute[ITI]{Institut für Theoretische Informatik}

\TitleImage[trim=-24mm -35mm 0 0,width=0.9\titleimagewd]{logo.pdf}

\begin{document}

\begin{frame}
\maketitle
\end{frame}

\section{Während der Entwurfsphase}

\begin{frame}
\frametitle{Die Arbeitsweise während der Entwurfsphase}

\begin{itemize}
    \item<+-> UML-Notation ist viel Aufwand und lässt sich nicht gut aktuell halten
    \item<+-> Keine Software mit Versionierung für UML-Modelle bekannt
    \item<+-> Sequenzdiagramme erlauben nur hinkend Entwurf von Abläufen und Interaktionen
    \item<+-> UML-Diagramme unterstützen beim Gesamtüberblick, insbesondere gemeinsam in der Gruppe
    \item<+-> UML-Entwurf fördert die Modualisierung und Anwendung von Entwurfsmustern, weil er nicht lauffähig sein muss
    \item<+-> Am besten Klassenstummel (mit Dokumentation) schreiben und Diagramme generieren lassen
\end{itemize}
\end{frame}

\section{Meilensteine der Entwurfsphase}

\begin{frame}
\frametitle{Die Meilensteine während der Entwurfsphase}

\begin{enumerate}
    \item<+-> Testprogramme in Worthwhile geschrieben und spezifiziert
    \item<+-> Sprachgrammatik festgelegt (Prototyp)
    \item<+-> Typsystem festgelegt (Prototyp)
    \item<+-> Transformation spezifischer Syntax und Semantik festgelegt
    \item<+-> Generiertes AST-Datenmodell in UML notiert
    \item<+-> Komponenten Interpreter und Beweiser in UML entworfen
    \item<+-> Eigenes AST-Datenmodell in UML entworfen
    \item<+-> Parser hinter Schnittstelle versteckt und Validator hinzugefügt
    \item<+-> Generierte GUI-Klassen in UML notiert
\end{enumerate}
\end{frame}

\section{Der Worthwhile-Entwurf}

\begin{frame}
\frametitle{Der Worthwhile-Entwurf}

\begin{itemize}
    \item<+-> Modell ist ein Worthwhile-Text aus einem beliebigen Readable-Objekt
    \item<+-> Parser erstellt mit Lexer AST-Sicht auf das Text-Modell
    \item<+-> AST-Sicht ist selbst Modell für Editor, Interpreter, Beweiser
    \item<+-> GUI ist Controller für Edit und Run, Debug, Prove
    \item<+-> GUI ist \textbf{nur} ein möglicher Controller und wird von Xtext als Eclipse-Plugin erstellt
\end{itemize}
\end{frame}

\subsection{AST}

\begin{frame}
\frametitle{Abstract-Syntax-Tree}

\begin{itemize}
    \item<+-> Kapselt bis auf Literalwerte ausschließlich Sprachsyntax
    \item<+-> Entkopplung der Semantik ohne spezialisierte Implementierungen
    \item<+-> Wendet Entwurfsmuster Besucher zur Modell-Verarbeitung an
\end{itemize}
\end{frame}

\subsection{Interpreter}

\begin{frame}
\frametitle{Interpreter}

\begin{itemize}
    \item<+-> Implementiert Besucher zur Ausführung und Auswertung
    \item<+-> Interaktion nach Ausführungsstart mit Beobachtern
    \item<+-> Keine dedizierte Run-time-Checker-Komponente, benutzt Beweiserschnittstelle selbst
\end{itemize}
\end{frame}

\subsection{Beweiserschnittstelle}

\begin{frame}
\frametitle{Beweiserschnittstelle}

\begin{itemize}
    \item<+-> Enthält getrennte Funktionseinheiten zur Transformation und Beweiserkommunikation
    \item<+-> Benutzt AST-Datenmodell zur internen Formelrepräsentation
    \item<+-> Implementiert Besucher zur Transformation und Übersetzung
\end{itemize}
\end{frame}

\section{Die Implementierungsphase}

\begin{frame}
\frametitle{Die Arbeitsweise während der Implementierungsphase}

\begin{itemize}
    \item<+-> Ausnutzung der Verfügbarkeit hilfreicher Werkzeuge wie Bug-Tracker, Code-Review, Continuous-Integration mit automatischen Tests und statischer Analyse
    \item<+-> AST-Attrappenerstellung für Testprogramme
    \item<+-> Parallele Komponentenentwicklung und ständige Integration durch Attrappenersetzung
\end{itemize}
\end{frame}
\end{document}
