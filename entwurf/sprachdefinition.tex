\section{Sprachdefinition}

\subsection{Trennung von Befehlen}

Die Trennung aufeinanderfolgender Befehle wird dadurch erreicht, dass in jeder Zeile höchstens ein Befehl stehen darf. Dies dient der Übersichtlichkeit, da hiermit Befehlstrenner wie das Semikolon unnötig werden.

\subsection{Operatorreihenfolge}

Die folgende Tabelle gibt die Reihenfolge an, in der Operatoren ausgewertet werden. Dabei werden die Operatoren mit höchster Priorität zuerst ausgewertet.

Operatoren mit gleicher Priorität werden von links nach rechts ausgewertet. Durch Klammersetzung kann die Auswertungsreihenfolge beeinflusst werden.

\begin{figure}[H]
\begin{tabular}{|l|l|}
\hline
\textbf{Priorität} & \textbf{Operatoren} \\
\hline
8 & \texttt{{[}\,{]}} \\
\hline
7 & $\neg, -, +$ (unäre Operatoren)\\
\hline
6 & $\cdot, \div, \%$\\
\hline
5 & $+, -$\\
\hline
4 & $<, \leq, \geq, >$\\
\hline
3 & $=$\\
\hline
2 & $\wedge$ \\
\hline
1 & $\vee$ \\
\hline
\end{tabular}
\end{figure}

\subsection{Formale Definition der Grammatik}

Ein Worthwhile-Programm besteht aus genau einem Element des Typs \textit{program}.

\setlength{\grammarindent}{12em} % increase separation between LHS/RH

\begin{grammar}

<program> ::= (<funcdecl> | <statement> | <axiom>)* 

<funcdecl> ::= `function' <type> <name> (<parameter>*) \\ (<requires> | <ensures>)* \\ <block>

<statement> ::= <vardecl> | <assign> | <block> | <annotation> | <funccall> | <conditional> | <loop> | <return>

<axiom> ::= `_axiom' (<expr> | <quantifiedexpr>)

<block> ::= `{' <statement>* `}'

<vardecl> ::= <parameter> `:=' <expr>

<assign> ::= <varref> `:=' <expr>

<varref> ::= <name> (`[' <expr> `]' )?

<annotation> ::= (`_assert' | `_assume') (<expr> | <quantifiedexpr>)

<funccall> ::= <name> `(' (<expr> (`,' <expr>)* )? `)'

<conditional> ::= `if' <expr> \\ <block>  \\ (`else' \\ <block>)?

<loop> ::= `while' <expr> \\ <invariant> \\ <block>

<invariant> ::= `_invariant' (<expr> | <quantifiedexpr>)

<return> ::= `return' <expr>

% TODO mark math symbols as tokens
<quantifiedexpr> ::= ($\forall$ | $\exists$ ) <type> <name> (`,' <expr>) (<quantifiedexpr> | `:' <expr>)

<expr> ::= <prefixop> <expr> | <expr> <binop> <expr> | <expr> <postfixop> | `(' <expr> `)' | <literal> | <varref> | <funccall>

<literal> ::= `false' | `true' | <integer>

<type> ::= <primitivetype> | <arraytype>

<primitivetype> ::= `Integer' | `Boolean'

<arraytype> ::= <primitivetype> `[' <expr> `]'

<parameter> ::= <type> <name>

<binop> ::= $\vee$ | $\wedge$ | $\textless$ | $\leq$ | $=$ | $\geq$ | $>$ | $+$ | $-$ | $\cdot$ | $/$

<prefixop> ::= $-$ | $+$ | $\neg$

<postfixop> ::= `[' <expr> `]'

<name> ::=  [`A--Za--z'][`A--Za--z0--9']*

\end{grammar}

\subsection{Reservierte Schlüsselwörter}

Folgende Schlüsselwörter sind für die Verwendung innerhalb der Sprache reserviert und dürfen daher nicht als Bezeichner für Variablen oder Funktionen verwendet werden:

\begin{itemize}
	\item \texttt{function}
	\item \texttt{Integer}
	\item \texttt{Boolean}
	\item \texttt{false}
	\item \texttt{true}	
	\item \texttt{\_axiom}
	\item \texttt{\_assert}
	\item \texttt{\_assume}
	\item \texttt{\_requires}
	\item \texttt{\_ensures}
	\item \texttt{if}
	\item \texttt{else}	
	\item \texttt{while}		
	\item \texttt{return}	
\end{itemize}

\subsection{Alternative Darstellungsweisen für mathematische Symbole}

Einige mathematische Symbole sind auf Tastaturen nicht verfügbar. Daher sind folgende textuelle Alternativen vorgesehen, um die Sprache auch ohne die Hilfe einer IDE editieren zu können:

\begin{figure}[H]
\begin{tabular}{|l|l|}
\hline
\textbf{Symbol} & \textbf{Alternative} \\
\hline
$\forall$ & \texttt{forall} \\
\hline
$\exists$ & \texttt{exists} \\
\hline
$\cdot$ & \texttt{*} \\
\hline
$\neg$ & \texttt{!} \\
\hline
$\vee$ & \texttt{\textbar\textbar} \\
\hline
$\wedge$ & \texttt{\&\&} \\
\hline
$\leq$ & \texttt{<=} \\
\hline
$\geq$ & \texttt{>=} \\
\hline
\end{tabular}
\end{figure}

\subsection{Typsystem}

Bedingt durch die kleine Anzahl möglicher Variablentypen ist das Typsystem nicht sehr komplex.

Jedes Sprachelement, das einen Wert zurückliefert, besitzt einen Rückgabetyp, kurz Typ des Sprachelements. Ein Sprachelement kann Kindelemente besitzen, deren Typ Einschränkungen unterliegen kann -- beispielsweise kann der Operator $+$ nur auf Operanden vom Typ \int{} angewendet werden.

In folgender Tabelle bedeutet die Notation $\rightarrow x$, dass der Typ des Sprachelements gleich dem Typ von $x$ ist.

\begin{landscape}

\enlargethispage{2cm}

\begin{figure}[H]
\begin{tabular}{lllp{11cm}}
\toprule
\textbf{Kategorie} & \textbf{Sprachelement} & \textbf{Typ} & \textbf{Einschränkungen} \\
\midrule
\multirow{4}{*}{Typdeklaration} & \int & \int & \\
\cmidrule{2-4}
& \bool & \bool & \\
\cmidrule{2-4}
& \multirow{2}{*}{$x$\texttt{[$N$]}} & \multirow{2}{*}{$\rightarrow x$} & $x \in \{$ \int{}, \bool{}$\}$ \\
& & & $N$ muss vom Typ \int{} sein \\
\midrule
Variablendeklaration & $type$ $name$ := $value$ & $\rightarrow type$ & Die Typen von $type$ und $value$ müssen gleich sein. \\
\midrule
\multirow{2}{*}{Variablenverweis} & $varref$ & $\rightarrow type$ & $type$ ist der in der Deklaration festgelegte Typ der Variablen. \\
\cmidrule{2-4}
& $varref$\texttt{[$index$]} & $\rightarrow varref$ & $index$ muss vom Typ \int{} sein. \\
\midrule
Variablenzuweisung & $varref$ := $value$ & $\rightarrow varref$ & Die Typen von $varref$ und $value$ müssen gleich sein. \\
\midrule
\multirow{6}{*}{Ausdrücke} & $x + y, x - y, x \cdot y, x \mathop{/} y, x \mathop{\%} y$ & \int & $x$ und $y$ müssen beide vom Typ \int{} sein. \\
\cmidrule{2-4}
& $a \wedge b, a \vee b$ & \bool & $a$ und $b$ müssen beide vom Typ \bool{} sein. \\
\cmidrule{2-4}
& $ -x, +x$ & \int & $x$ muss vom Typ \int{} sein. \\
\cmidrule{2-4}
& $\neg a$ & \bool & $a$ muss vom Typ \bool{} sein. \\
\cmidrule{2-4}
& \multirow{2}{*}{$x = y$} & \multirow{2}{*}{\bool} & $x$ und $y$ müssen beide vom gleichen Typ sein. \\
& & & $x$ und $y$ müssen vom Typ \int{} oder \bool{} sein. \\
\cmidrule{2-4}
& $x \leq y, x < y, x > y, x \geq y$ & \bool & $x$ und $y$ müssen beide vom Typ \int{} sein. \\
\midrule
Funktionsdeklaration & \texttt{function} $type$ $name$ ($params$) & $\rightarrow type$ & \\
\midrule
Funktionsverweis & $funcref$ & $\rightarrow type$ & $type$ ist der in der Deklaration festgelegte Rückgabetyp der Funktion. \\
\midrule
Funktionsaufruf & $funcref(params)$ & $\rightarrow funcref$ & Die Parameter $params$ müssen in Anzahl und jeweiligem Typ den deklarierten Parametern entsprechen. \\
\midrule
Rückgabewert & \texttt{return} $value$ & $-$ & Der Typ von $value$ muss gleich dem Rückgabetyp der umschließenden Funktion sein. \\
\bottomrule
\end{tabular}
\end{figure}
\end{landscape}
