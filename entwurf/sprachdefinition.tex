\section{Sprachdefinition}

\subsection{Trennung von Befehlen}

Die Trennung aufeinanderfolgender Befehle wird dadurch erreicht, dass in jeder Zeile höchstens ein Befehl stehen darf. Dies dient der Übersichtlichkeit, da hiermit Befehlstrenner wie das Semikolon unnötig werden.

\subsection{Operatorreihenfolge}

Die folgende Tabelle gibt die Reihenfolge an, in der Operatoren ausgewertet werden. Dabei werden die Operatoren mit höchster Priorität zuerst ausgewertet.

Operatoren mit gleicher Priorität werden von links nach rechts ausgewertet. Durch Klammersetzung kann die Auswertungsreihenfolge beeinflusst werden.

\begin{figure}[H]
\begin{tabular}{|l|l|}
\hline
\textbf{Priorität} & \textbf{Operatoren} \\
\hline
8 & \texttt{{[}\,{]}} \\
\hline
7 & $\neg, -, +$ (unäre Operatoren)\\
\hline
6 & $\cdot, \div, \%$\\
\hline
5 & $+, -$\\
\hline
4 & $<, \leq, \geq, >$\\
\hline
3 & $=$\\
\hline
2 & $\wedge$ \\
\hline
1 & $\vee$ \\
\hline
\end{tabular}
\end{figure}

\subsection{Formale Definition der Grammatik}

Ein Worthwhile-Programm besteht aus genau einem Element des Typs \textit{program}.

\setlength{\grammarindent}{12em} % increase separation between LHS/RH

\begin{grammar}

<program> ::= (<funcdecl> | <statement> | <axiom>)* 

<funcdecl> ::= `function' <type> <name> `('<parameter>*`)' \\ (<requires> | <ensures>)* \\ <block>

<statement> ::= <vardecl> | <assign> | <block> | <annotation> | <funccall> | <conditional> | <loop> | <return>

<axiom> ::= `_axiom' <quantifiedexpr>

<requires> ::= `_requires' <predformula>

<ensures> ::= `_ensures' <predformula>

<block> ::= `{' <statement>* `}'

<vardecl> ::= <type> <name> ( `:=' <expr> )?

<assign> ::= <varref> `:=' <expr>

<varref> ::= <name> (`[' <expr> `]' )?

<annotation> ::= (`_assert' | `_assume') <predformula>

<funccall> ::= <name> `(' (<expr> (`,' <expr>)* )? `)'

<conditional> ::= `if' <expr> \\ <block>  \\ (`else' \\ <block>)?

<loop> ::= `while' <expr> \\ <invariant> \\ <block>

<invariant> ::= `_invariant' <predformula>

<return> ::= `return' <expr>

<predformula> ::= <expr> | <quantifiedexpr>

<quantifiedexpr> ::= (\lit{$\forall$} | \lit{$\exists$} ) <type> <name> (`,' <expr>) (<quantifiedexpr> | `:' <expr>)

<expr> ::= <prefixop> <expr> | <expr> <binop> <expr> | <expr> <postfixop> | `(' <expr> `)' | <arraylen> | <literal> | <varref> | <funccall>

<arraylen> ::= `length' `(' <expr> `)'

<literal> ::= `false' | `true' | <integer>

<type> ::= <primitivetype> | <arraytype>

<primitivetype> ::= `Integer' | `Boolean'

<arraytype> ::= <primitivetype> `[' <expr> `]'

<parameter> ::= <type> <name>

<binop> ::= $\vee$ | $\wedge$ | $\textless$ | $\leq$ | $=$ | $\geq$ | $>$ | $+$ | $-$ | $\cdot$ | $/$

<prefixop> ::= $-$ | $+$ | $\neg$

<postfixop> ::= `[' <expr> `]'

<name> ::=  [`A--Za--z'][`A--Za--z0--9']*

\end{grammar}

\subsection{Reservierte Schlüsselwörter}

Folgende Schlüsselwörter sind für die Verwendung innerhalb der Sprache reserviert und dürfen daher nicht als Bezeichner für Variablen oder Funktionen verwendet werden:

\begin{itemize}
	\item \texttt{function}
	\item \texttt{Integer}
	\item \texttt{Boolean}
	\item \texttt{false}
	\item \texttt{true}	
	\item \texttt{\_axiom}
	\item \texttt{\_assert}
	\item \texttt{\_assume}
	\item \texttt{\_requires}
	\item \texttt{\_ensures}
	\item \texttt{if}
	\item \texttt{else}	
	\item \texttt{while}		
	\item \texttt{return}	
\end{itemize}

\subsection{Semantische Einschränkungen}
Aus der Semantik der Sprache ergeben sich weitere Einschränkungen, die nicht in der Sprachdefinition abgebildet sind:

\begin{itemize}
	\item Eine Variable kann nur in Anweisungen referenziert werden, die nach der Deklaration stehen (insbesondere also nicht in der Deklaration selbst).
	\item Eine Funktion kann nur in Anweisungen referenziert werden, die nach der Funktionsdefintion stehen (insbesondere also nicht in der Funktion selbst). Damit wird effektiv die Verwendung von Rekursion verunmöglicht.
	\item Eine \textit{return}-Anweisung kann nur innerhalb einer Funktion stehen.
	\item Eine Funktion muss einen Rückgabewert haben; es müssen also alle möglichen Codepfade innerhalb der Funktion eine \textit{return}-Anweisung beinhalten.
	\item Nach Ausführung der \textit{return}-Anweisung wird die Funktion verlassen und keine weiteren Anweisungen der Funktion werden ausgeführt.
	\item Wenn bei der Deklaration einer Variablen kein Wert festgelegt wird, wird ein Standardwert angenommen. Dieser ist 0 für Variablen vom Type \int{}, \texttt{false} für Variablen vom Typ \bool{} und ein Array mit Standardwerten des jeweiligen Typs für Arrays.
\end{itemize}

\subsection{Alternative Darstellungsweisen für mathematische Symbole}
\label{textrep}
Einige mathematische Symbole sind auf Tastaturen nicht verfügbar. Daher sind folgende textuelle Alternativen vorgesehen, um die Sprache auch ohne die Hilfe einer IDE editieren zu können:

\begin{figure}[H]
\begin{tabular}{|l|l|}
\hline
\textbf{Symbol} & \textbf{Alternative} \\
\hline
$\forall$ & \texttt{forall} \\
\hline
$\exists$ & \texttt{exists} \\
\hline
$\cdot$ & \texttt{*} \\
\hline
$\neg$ & \texttt{!} \\
\hline
$\vee$ & \texttt{\textbar\textbar} \\
\hline
$\wedge$ & \texttt{\&\&} \\
\hline
$\leq$ & \texttt{<=} \\
\hline
$\geq$ & \texttt{>=} \\
\hline
$\neq$ & \texttt{!=} \\
\hline
\end{tabular}
\end{figure}

\subsection{Typsystem}

Bedingt durch die kleine Anzahl möglicher Variablentypen ist das Typsystem nicht sehr komplex.

Jedes Sprachelement, das einen Wert zurückliefert, besitzt einen Rückgabetyp, kurz Typ des Sprachelements. Ein Sprachelement kann Kindelemente besitzen, deren Typ Einschränkungen unterliegen kann -- beispielsweise kann der Operator $+$ nur auf Operanden vom Typ \int{} angewendet werden.

Die Spezifikation des Typsystems ist in Tabelle~\ref{typesystem} dargestellt.

\subsection{Scoping}

Um die Transformierbarkeit der Sprache in eine prädikatenlogische Formel zu erleichtern, werden einige Einschränkungen bezüglich der Sichtbarkeit von Variablen und Funktionen vorgenommen.

Im Einzelnen gibt es folgende Gültigkeitsbereiche (Scopes) in einem Programm:

\begin{description}
	\item[Globaler Gültigkeitsbereich] In diesem Gültigkeitsbereich befinden sich alle Variablen, die nicht innerhalb einer Funktion deklariert wurden.
	\item[Gültigkeitsbereich einer Funktion] Jede Funktion besitzt ihren eigenen Gültigkeitsbereich. Hier sind nur sichtbar: \begin{itemize}
		\item In der Funktion deklarierte Variablen
		\item Parameter der Funktion
	\end{itemize}
	Insbesondere sind keine im globalen Gültigkeitsbereich oder in aufrufenden Funktionen deklarierten Variablen sichtbar.
	\item[Axiome] Da Axiome globale Gültigkeit haben sollen, sind in einem Axiom keine Variablen sichtbar. Variablen können also nur über Quantorenausdrücke angegeben werden.
	\item[Parameter einer Funktion] Die Parameterliste einer Funktion bildet einen eigenen Gültigkeitsbereich. Die einzigen sichtbaren Variablen sind andere Parameter.
\end{description}



\begin{landscape}

\begin{longtable}{lllp{11cm}}
\label{typesystem} \\
\caption{Typsystem-Spezifikation. Die Notation $\rightarrow x$ bedeutet, dass der Typ des Sprachelements gleich dem Typ von $x$ ist.} \\
\toprule
\textbf{Kategorie} & \textbf{Sprachelement} & \textbf{Typ} & \textbf{Einschränkungen} \\
\midrule

\endhead

\endfoot

\multirow{4}{*}{Typdeklaration} & \int & \int & \\
\cmidrule{2-4}
& \bool & \bool & \\
\cmidrule{2-4}
& \multirow{2}{*}{$x$\texttt{[$N$]}} & \multirow{2}{*}{$\rightarrow x$} & $x \in \{$ \int{}, \bool{}$\}$ \\
& & & $N$ muss vom Typ \int{} sein \\
\midrule
Variablendeklaration & $type$ $name$ := $value$ & -- & Die Typen von $type$ und $value$ müssen gleich sein. \\
\midrule
\multirow{2}{*}{Variablenverweis} & $varref$ & $\rightarrow type$ & $type$ ist der in der Deklaration festgelegte Typ der Variablen. \\
\cmidrule{2-4}
& $varref$\texttt{[$index$]} & $\rightarrow varref$ & $index$ muss vom Typ \int{} sein. \\
\midrule
Variablenzuweisung & $varref$ := $value$ & $\rightarrow varref$ & Die Typen von $varref$ und $value$ müssen gleich sein. \\
\midrule
\multirow{7}{*}{Ausdrücke} & $x + y, x - y, x \cdot y, x \mathop{/} y, x \mathop{\%} y$ & \int & $x$ und $y$ müssen beide vom Typ \int{} sein. \\
\cmidrule{2-4}
& $a \wedge b, a \vee b$ & \bool & $a$ und $b$ müssen beide vom Typ \bool{} sein. \\
\cmidrule{2-4}
& $ -x, +x$ & \int & $x$ muss vom Typ \int{} sein. \\
\cmidrule{2-4}
& $\neg a$ & \bool & $a$ muss vom Typ \bool{} sein. \\
\cmidrule{2-4}
& \multirow{2}{*}{$x = y$} & \multirow{2}{*}{\bool} & $x$ und $y$ müssen beide vom gleichen Typ sein. \\
& & & $x$ und $y$ müssen vom Typ \int{} oder \bool{} sein. \\
\cmidrule{2-4}
& $x \leq y, x < y, x > y, x \geq y$ & \bool & $x$ und $y$ müssen beide vom Typ \int{} sein. \\
\cmidrule{2-4}
& $\forall$ $x$ $expr$ und $\exists$ $x$ $expr$ & \bool & $x$ muss vom Typ \int{} oder \bool{} sein.\\
\midrule
Funktionsverweis & $funcref$ & $\rightarrow type$ & $type$ ist der in der Deklaration festgelegte Rückgabetyp der Funktion. \\
\midrule
Funktionsaufruf & $funcref(params)$ & $\rightarrow funcref$ & Die Parameter $params$ müssen in Anzahl und jeweiligem Typ den deklarierten Parametern entsprechen. \\
\midrule
Rückgabewert & \texttt{return} $value$ & $-$ & Der Typ von $value$ muss gleich dem Rückgabetyp der umschließenden Funktion sein. \\
\midrule
Arraygröße & \texttt{length(} $expr$ \texttt{)} & \int & $expr$ muss ein Array sein. \\
\midrule
Verzweigung & \texttt{if} $expr$ \ldots & $-$ & $expr$ muss vom Typ \bool{} sein. \\
\midrule
Schleife & \texttt{while} $expr$ \ldots & $-$ & $expr$ muss vom Typ \bool{} sein. \\
\midrule
Annotation & $\begin{rcases} \text{\texttt{\_axiom}} \\ \text{\texttt{\_assert}} \\ \text{\texttt{\_assume}} \\ \text{\texttt{\_requires}} \\ \text{\texttt{\_ensures}} \end{rcases} expr \ldots$ & $-$ & $expr$ muss vom Typ \bool{} sein. \\
\bottomrule
\end{longtable}

\end{landscape}
