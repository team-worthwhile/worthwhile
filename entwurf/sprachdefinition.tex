\section{Sprachdefinition}

\subsection{Operatorreihenfolge}

Die folgende Tabelle gibt die Reihenfolge an, in der Operatoren ausgewertet werden. Dabei werden die Operatoren mit höchster Priorität zuerst ausgewertet.

Operatoren mit gleicher Priorität werden von links nach rechts ausgewertet. Durch Klammersetzung kann die Auswertungsreihenfolge beeinflusst werden.

\begin{figure}[H]
\begin{tabular}{|l|l|}
\hline
\textbf{Priorität} & \textbf{Operatoren} \\
\hline
8 & \texttt{{[}\,{]}} \\
\hline
7 & $\neg, -, +$ (unäre Operatoren)\\
\hline
6 & $\cdot, \div, \%$\\
\hline
5 & $+, -$\\
\hline
4 & $<, \leq, \geq, >$\\
\hline
3 & $=$\\
\hline
2 & $\wedge$ \\
\hline
1 & $\vee$ \\
\hline
\end{tabular}
\end{figure}

\subsection{Formale Definition der Grammatik}

\setlength{\grammarparsep}{20pt plus 1pt minus 1pt} % increase separation between rules
\setlength{\grammarindent}{12em} % increase separation between LHS/RH

\begin{grammar}

<statement> ::= <ident> `=' <expr> 
\alt `for' <ident> `=' <expr> `to' <expr> `do' <statement> 
\alt `{' <stat-list> `}' 
\alt <empty> 

<stat-list> ::= <statement> `;' <stat-list> | <statement> 

\end{grammar}
