\section{Glossar}

% TODO: alphabetisch sortieren (wenn Glossar fertig)

\begin{description}
    \item[Befehl (GUI)] Aktion, die der Benutzer durch Aufruf eines Menüeintrags, Anklicken einer Schaltfläche oder Betätigung einer Tastenkombination auslösen kann.
    \item[Programm] Folge von Anweisungen formuliert nach den Regeln einer Programmiersprache, die vom Computer ausgeführt werden können.
    \item[Programmzustand] Eindeutige Beschreibung des Zustandes eines in Ausführung befindlichen Programms, bestehend aus Anweisungszeiger, Aufrufstack und Variablenwerten.
    \item[Parser] Programm, das aus einem Quelltext in Textform eine maschinenlesbare Repräsentation dieses Quelltextes erstellt.
    \item[Interpreter] Ein Programm, welches Anweisungen in maschinenlesbarer Quelltextform in Anweisungen für den Computer umsetzt und diese ausführt.
    \item[Run-time checker] Eine Komponente der Laufzeitumgebung, der zur Laufzeit das Zutreffen der Zusicherungen auf den Programmzustand überprüft und eventuelle Fehler meldet.
    \item[Beweiser] Ein Programm, mit dessen Hilfe man die Gültigkeit von Zusicherungen für ein Programm für alle Programmeingaben beweisen kann.
    \item[Zusicherung] Anforderung an den Programmzustand, die an einer oder mehreren festgelegten Stellen im Programmablauf gelten muss.
    \item[Annotation] Syntaxelement einer Programmiersprache, das selbst nicht vom Interpreter ausgeführt wird, sondern zusätzliche Informationen über ein anderes Syntaxelement (eventuell für ein externes System) bereitstellt.
    %\item[Beweisverpflichtung] TODO: What is this?
    \item[Prädikatenlogik] Ein logisches System, das es erlaubt, mathematische Aussagen zu formalisieren und auf Gültigkeit zu überprüfen. %First-order, Second-order
    \item[WHILE-Sprache (Interpreter)] Einfache Programmiersprache, in der nur simple Syntaxkonstrukte wie z.~B.\ eine \texttt{while}-Schleife verfügbar sind.
    \item[Annotationssprache (Beweiser)] Eine Sprache, in der \see Zusicherungen formuliert werden. Sie kann als eine Annotation direkt an der passenden Stelle eines Programms eingefügt werden.
    \item[Schleifeninvariante] Aussage, die beim Eintritt in einer Schleife, bei jedem Schleifendurchlauf und beim Verlassen der Schleife erfüllt ist.
    \item[Gültigkeitsbereich] Eindeutig umgrenzter Bereich in einem Programmcode, beispielsweise der Rumpf einer Schleife oder einer Methode. Auf eine dort deklarierte Variable oder Annahme kann nur in diesem Bereich Bezug genommen werden.
    \item[Zusicherungensprache (Run-time checker)]
    \item[SMTLIB2] (Satisfiability Modulo Theories Library): Format zur Spezifikation von prädikatenlogischen Formeln zwecks Überprüfung auf Erfüllbarkeit, erstellt von der gleichnamigen Initiative
    \item[Z3] Von Microsoft Research entwickeltes Computerprogramm zur Erfüllbarkeitsüberprüfung prädikatenlogischer Formeln
    \item[Eclipse] Java-Entwicklungsumgebung basierend auf einer leicht erweiterbaren Plugin-Architektur.
    \item[Whitespace] Mit Leerzeichen, Tabulatorzeichen oder Zeilenumbrüchen gefüllter und damit ``leerer'' Bereich im Quelltext.
\end{description}

