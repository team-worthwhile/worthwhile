\section{Glossar}
\newcommand{\glossaritem}[1]{\item[#1]}

% TODO: alphabetisch sortieren (wenn Glossar fertig)

\begin{description}
    \label{Zusicherung}
    \glossaritem{Annotation} Syntaxelement einer Programmiersprache, das selbst nicht vom Interpreter ausgeführt wird, sondern zusätzliche Informationen über ein anderes Syntaxelement (eventuell für ein externes System) bereitstellt.
    \glossaritem{Annotationssprache (Beweiser)} Eine Sprache, in der Zusicherungen formuliert werden. Sie kann als eine Annotation direkt an der passenden Stelle eines Programms eingefügt werden.
    \glossaritem{Befehl~(GUI)} Aktion, die der Benutzer durch Aufruf eines Menüeintrags, Anklicken einer Schaltfläche oder Betätigung einer Tastenkombination auslösen kann.
    \glossaritem{Beweiser} Ein Programm, mit dessen Hilfe man die Gültigkeit von Zusicherungen für ein Programm für alle Programmeingaben beweisen kann.
    %\glossaritem[Beweisverpflichtung] TODO: What is this?
    \glossaritem{Eclipse} Java-Entwicklungsumgebung basierend auf einer leicht erweiterbaren Plugin-Architektur.
    \glossaritem{Gültigkeitsbereich} Eindeutig umgrenzter Bereich in einem Programmcode, beispielsweise der Rumpf einer Schleife oder einer Methode. Auf eine dort deklarierte Variable oder Annahme kann nur in diesem Bereich Bezug genommen werden.
    \glossaritem{Interpreter} Ein Programm, welches Anweisungen in maschinenlesbarer Quelltextform in Anweisungen für den Computer umsetzt und diese ausführt.
    \glossaritem{Parser} Programm, das aus einem Quelltext in Textform eine maschinenlesbare Repräsentation dieses Quelltextes erstellt.
    \glossaritem{Prädikatenlogik} Ein logisches System, das es erlaubt, mathematische Aussagen zu formalisieren und auf Gültigkeit zu überprüfen. %First-order, Second-order
    \glossaritem{Programm} Folge von Anweisungen formuliert nach den Regeln einer Programmiersprache, die vom Computer ausgeführt werden können.
    \glossaritem{Programmzustand} Eindeutige Beschreibung des Zustandes eines in Ausführung befindlichen Programms, bestehend aus Anweisungszeiger, Aufrufstack und Variablenwerten.
    \glossaritem{Run-time-checker} Eine Komponente der Laufzeitumgebung, die zur Laufzeit das Zutreffen der Zusicherungen auf den Programmzustand überprüft und eventuelle Fehler meldet.
    \glossaritem{Schleifeninvariante} Aussage, die beim Eintritt in einer Schleife, bei jedem Schleifendurchlauf und beim Verlassen der Schleife erfüllt ist.
    \glossaritem{SMT-LIB~(Satisfiability Modulo Theories Library)} Format zur Spezifikation von prädikatenlogischen Formeln zwecks Überprüfung auf Erfüllbarkeit, erstellt von der gleichnamigen Initiative.
    \glossaritem{WHILE-Sprache (Interpreter)} Einfache Programmiersprache, in der nur simple Syntaxkonstrukte wie z.~B.\ eine \texttt{while}-Schleife verfügbar sind.
    \glossaritem{Z3} Von Microsoft Research entwickeltes Computerprogramm zur Erfüllbarkeitsüberprüfung prädikatenlogischer Formeln.
    \glossaritem{Zusicherung} Anforderung an den Programmzustand, die an einer oder mehreren festgelegten Stellen im Programmablauf gelten muss.
    \glossaritem{Zusicherungensprache (Run-time-checker)} Eine Sprache, in der Zusicherungen für den \see{Run-time-checker} formuliert sind.
    \glossaritem{Continous Integration} Ein Entwicklungsmodell, bei dem durch kontinuierliche Qualitätskontrolle, beispielsweise durch automatische Tests, schon während der Entwicklung eine möglichst hohe Code- und Gesamtqualität der Software erreicht werden soll.
\end{description}

