\section{Glossar}
\begin{description}
    \item[Befehl (GUI)] Aktion, die der Benutzer durch Aufruf eines Menüeintrags, Anklicken einer Schaltfläche oder Betätigung einer Tastenkombination auslösen kann.
    \item[Programm]
    \item[Programmzustand] Eindeutige Beschreibung des Zustandes eines in Ausführung befindlichen Programms, bestehend aus Anweisungszeiger, Aufrufstack und Variablenwerten.
    Parser: Programm, das aus einem Quelltext in Textform eine maschinenlesbare Repräsentation dieses Quelltextes erstellt.
    \item[Interpreter]
    \item[Run-time checker]
    \item[Beweiser]
    \item[Zusicherung]
    \item[Annotation] Syntaxelement einer Programmiersprache, das selbst nicht vom Interpreter ausgeführt wird, sondern zusätzliche Informationen über ein anderes Syntaxelement (eventuell für ein externes System) bereitstellt.
    \item[Beweisverpflichtung]
    \item[Prädikatenlogik] First-order, Second-order
    \item[WHILE-Sprache (Interpreter)]
    \item[Annotationssprache (Beweiser)]
    \item[Schleifeninvariante] Aussage, die beim Eintritt in einer Schleife, bei jedem Schleifendurchlauf und beim Verlassen der Schleife erfüllt ist.
    \item[Gültigkeitsbereich] Eindeutig umgrenzter Bereich in einem Programmcode, beispielsweise der Rumpf einer Schleife oder einer Methode. Auf eine dort deklarierte Variable oder Annahme kann nur in diesem Bereich Bezug genommen werden.
    \item[Zusicherungensprache (Run-time checker)]
    \item[SMTLIB2] (Satisfiability Modulo Theories Library): Format zur Spezifikation von prädikatenlogischen Formeln zwecks Überprüfung auf Erfüllbarkeit, erstellt von der gleichnamigen Initiative
    \item[Z3] Von Microsoft Research entwickeltes Computerprogramm zur Erfüllbarkeitsüberprüfung prädikatenlogischer Formeln
    \item[Eclipse] Java-Entwicklungsumgebung basierend auf einer leicht erweiterbaren Plugin-Architektur.
    \item[Online-Dokumentation] Programmdokumentation, die im Gegensatz zum gedruckten Handbuch als elektronisches Dokument verfügbar ist und aus dem Produkt heraus aufgerufen werden kann.
\end{description}

