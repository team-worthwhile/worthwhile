\section{Produkteinsatz}%

\subsection{Anwendungsbereich}%

\begin{itemize}%
    \item Die Anwendung dient zur Analyse und \see Verifikation von Programmen und soll Anwendern speziell im Kontext der Forschung und Lehre die M"oglichkeit geben, schnell eigene Ideen in eine Programmiersprache zu fassen und mit einem \see Beweiser zu analysieren.%
    \item Der interaktive Prozess zwischen Benutzer und Beweiser (über die Schnittstelle Worthwhile), wenn die Überprüfung der Programm- bzw. Funktionsspezifikation durch den Beweiser fehlschlägt und die Spezifikation (bei korrekter Implementierung) angepasst werden muss, führt schließlich zur einer verifizierbaren Spezifikation und gleichzeitig zu einem besseren Verständnis des Entwicklers, was das Programm bzw. eine Funktion bewiesen tut. Nicht zuletzt liefert die Menge der Spezifikationen eine gute Dokumentation von Programmierschnittstellen.%
\end{itemize}%

\subsection{Zielgruppe}%

\begin{itemize}%
    \item Die Anwendung soll von Lernenden sowie Lehrenden im Umfeld einer Lehreinrichtung (z.~B.\ Besucher von Einf"uhrungsveranstaltungen in die Programmverifikation) verwendet werden. Entsprechend kann man Kenntnis von anderen Programmiersprachen sowie Vertrautheit mit Grundkonzepten der Programmverifikation (z.~B.\ der Beschreibung eines \see Programmzustandes mit \see Quantoren) voraussetzen.%
\end{itemize}%

\subsection{Betriebsbedingungen}%

\begin{itemize}%
    \item \see Java-f"ahiger PC mit Farbbildschirm, empfohlene Bildschirmaufl"osung mindestens 1024$\times$768%
    \item Auf dem PC muss gen"ugend Speicherplatz ($\ge$ 200MiB) verf"ugbar sein.%
    \item Auf dem PC muss gen"ugend Arbeitsspeicher ($\ge$ 512 MB) verf"ugbar sein.%
\end{itemize}%
