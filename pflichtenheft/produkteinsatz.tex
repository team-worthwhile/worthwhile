\section{Produkteinsatz}%

\subsection{Anwendungsbereich}%

Die Anwendung dient zur Analyse und \see Verifikation von Programmen und soll Anwendern speziell im Kontext der Forschung und Lehre die M"oglichkeit geben, schnell eigene Ideen in eine Programmiersprache zu fassen und mit einem \see Beweiser zu analysieren.%

Außerdem dient die Anwendung als Schnittstelle im interaktiven Prozess zwischen Benutzer und Beweiser, der entsteht, wenn die Überprüfung der Programm- bzw. Funktionsspezifikation durch den Beweiser fehlschlägt und die Spezifikation (bei korrekter Implementierung) angepasst werden muss. Dies führt schließlich zur einer verifizierbaren Spezifikation und gleichzeitig zu einem besseren Verständnis des Entwicklers, was das Programm bzw. eine Funktion bewiesenermaßen tut. Nicht zuletzt liefert die Menge der Spezifikationen eine gute Dokumentation von Programmierschnittstellen.%

\subsection{Zielgruppe}%

Die Anwendung soll von Lernenden sowie Lehrenden im Umfeld einer Lehreinrichtung (z.~B.\ Besucher von Einf"uhrungsveranstaltungen in die Programmverifikation) verwendet werden. Entsprechend kann man Kenntnis von anderen Programmiersprachen sowie Vertrautheit mit Grundkonzepten der Programmverifikation (z.~B.\ der Beschreibung eines \see Programmzustandes mit \see Quantoren) voraussetzen.%

\subsection{Betriebsbedingungen}%

\textbf{TODO!}
