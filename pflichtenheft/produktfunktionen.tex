\section{Funktionale Anforderungen}

Die funktionalen Anforderungen sind abschnittsweise absteigend nach ihrer Wichtigkeit sortiert.

\subsection{WHILE-""Sprache}

Ein vom Benutzer eingegebenes Programm muss den Spezifikationen der im folgenden definierten WHILE-""Sprache genügen.
% TODO Die Kriterien sind in jedem Abschnitt jeweils nach absteigender Priorität geordnet; dabei sind Musskriterien unbedingt einzuhalten, Sollkriterien wünschenswert und Wunschkriterien optionale Verbesserungen.
% TODO Musskriterien
	\begin{reqlist}{FW}
	    	\req{010}{Variablen}{Der Benutzer hat die Möglichkeit, Variablen zu deklarieren und ihnen einen Wert zuzuweisen. Als Variablentypen werden Ganzzahlen und Wahrheitswerte (Booleans) unterstützt, weiterhin Arrays von Variablen eines Typs.}
		\req{020}{Bedingte Anweisungen}{Die Sprache unterstützt bedingte Anweisungen, mit denen ein Codeabschnitt nur bei erfüllter Bedingung ausgeführt wird.}
		\req{030}{Schleifen}{Die Sprache unterstützt das Sprachkonstrukt der \texttt{while}-""Schleife, mit der ein Codepfad ausgeführt wird, solange eine Bedinung erfüllt ist. Die Bedingung wird dabei vor jedem Schleifendurchlauf überprüft.}
		\req{040}{Mathematische und logische Operatoren}{Die Sprache unterstützt die gängigen mathematischen Operatoren $+$~(Addition), $-$~(Subtraktion), $*$~(Multiplikation), $/$~(Ganzzahlige Division), \^{ }~(Potenz) und $\%$~(Modulo) sowie die logischen Operatoren $\wedge$~(Konjunktion), $\vee$~(Disjunktion) und $\neg$~(Negation). Es werden außerdem die Vergleichsoperatoren $=$, $\ne$, $<$, $>$, $\leq$ und $\geq$ unterstützt. Zusätzlich werden die Booleschen Konstanten \texttt{true}~(wahr) und \texttt{false}~(falsch) unterstützt.}
		\req{060}{Kommentare}{Die Sprache unterstützt Kommentare, die dem Benutzer zum Verständnis des Codes dienen und vom Interpreter ignoriert werden.}
	\end{reqlist}
% TODO Sollkriterien
	\begin{reqlist}{FW}
		\req{050}{Funktionen}{Die Sprache unterstützt die Definition und das Aufrufen benutzerdefinierter Funktionen. Eine Funktion kann optional Parameter entgegennehmen und hat einen Rückgabewert eines bestimmten Typs.}
	\end{reqlist}


\subsection{Annotationssprache}

% TODO Musskriterien
	\begin{reqlist}{FA}
		\req{010}{Zusicherungen}{Die Annotationssprache lässt die Angabe von Zusicherungen zu, die vom Run-time-""Checker oder vom Beweiser überprüft werden sollen.}
		\req{020}{Annahmen}{Die Annotationssprache lässt die Angabe von Annahmen zu, die bei der Überprüfung von Aussagen im selben Gültigkeitsbereich als wahr vorausgesetzt werden.}
		\req{060}{Prädikatenlogische Formeln}{Aussagen und Annahmen können als prädikatenlogische Formeln erster Stufe mit den Quantoren $\forall$~(Allquantor) und $\exists$~(Existenzquantor) formuliert werden. Dabei wird die Syntax von Ausdrücken aus der WHILE-""Sprache übernommen und um die Angabe von Quantoren erweitert.}
	\end{reqlist}
% TODO Sollkriterien
	\begin{reqlist}{FA}
		\req{030}{Axiome}{Die Annotationssprache lässt die Angabe von Axiomen (globale Annahmen) zu, die in allen Gültigkeitsbereichen als wahr vorausgesetzt werden.}
		\req{040}{Invarianten}{Die Sprache lässt die Angabe von Aussagen über Schleifeninvarianten zu.}
		\req{050}{Vor- und Nachbedingungen}{Die Sprache lässt die Angabe von Vor- und Nachbedingungen für Funktionen zu.}%
	\end{reqlist}



\subsection{Interpreter}

% TODO Musskriterien

	Der \see{Interpreter} kann ein Programm in einem von zwei verschiedenen Modi ausf"{u}hren:

	\begin{itemize}
		\item
		\begin{reqlist}{FI}
		\req{010}{Debug-""Modus}{Hier wird das Programm schrittweise ausgef\"{u}hrt und bei Erreichen eines Breakpoints die Programmausf\"{u}hrung pausiert.}
		\end{reqlist}

		\item
		\begin{reqlist}{FI}
		\req{020}{Ausf\"{u}hrungsmodus}{Hier wird das Programm ausgef\"{u}hrt, ohne dass der Benutzer die M\"{o}glichkeit zur Pausierung der Programmausf"uhrung hat.}
		\end{reqlist}
	\end{itemize}

	\begin{reqlist}{FI}
		\req{030}{Abbruch einer Programmausf"uhrung}{Ein laufendes oder pausiertes Programm kann jederzeit abgebrochen werden, womit der Interpreter in den Zustand vor Ausf\"{u}hrung des Programms zur"uckversetzt wird.}
	\end{reqlist}%

Ist eine Programmausf"uhrung nach Erreichen eines Breakpoints im Debug-""Modus unterbrochen, sind folgende Debugger-""Funktionen verf"ugbar:%

	\begin{reqlist}{FI}%
		\req{040}{Einzelschritt}{Die n\"{a}chste Anweisung wird ausgef\"{u}hrt und anschlie"send wird die Ausf\"{u}hrung des Programms pausiert. Wenn die ausgef"uhrte Anweisung ein Funktionsaufruf war, wird im n"achsten Schritt mit der ersten Anweisung der aufgerufenen Funktion fortgefahren.}
	\end{reqlist}
% TODO Sollkriterien
	\begin{reqlist}{FI}
		\req{060}{Ausdrucksauswertung}{Der Interpreter kann bei pausierter Programmausf"uhrung einen Ausdruck interpretieren und auswerten, der im aktuellen Zustand des Programms g\"{u}ltig ist. Parserfehler in einem Ausdruck werden erkannt und ausgegeben.}
	\end{reqlist}
% TODO Wunschkriterien
		\begin{reqlist}{FI}
			\req{050}{\"{U}berspringen}{Die n\"{a}chste Anweisung wird ausgef\"{u}hrt und anschlie"send wird die Ausf\"{u}hrung des Programms pausiert. Wenn die ausgef"uhrte Anweisung ein Funktionsaufruf war, wird die aufgerufene Funktion entweder bis zum Erreichen eines Breakpoints oder einschlie"slich des R"ucksprungs ausgef"uhrt.}
		\end{reqlist}

\subsection{Run-time-""Checker}%

Zur Prüfung von Annotationen durch den Run-time-""Checker gibt der Programmzustand die festen Belegungen der in den Annotationsaussagen referenzierten Variablen vor.%

\begin{reqlist}{FR}%
    \req{010}{Prüfung von Zusicherungen}{Eine Zusicherung ist genau dann erfüllt, wenn ihre Aussage über den Programmzustand wahr ist.}%
    \req{020}{Prüfung von Annahmen}{Annahmen werden als Zusicherungen ausgewertet.}%
    \req{030}{Prüfung von Aussagen über uneingeschränktem Bereich}{Wenn Formeln über uneingeschränktem Definitionsbereich ausgewertet werden, werden diese dem Beweiser zusammen mit der Programmspezifikation und dem -zustand übergeben.}%
\end{reqlist}%

\subsection{Beweiserschnittstelle}%

	\begin{reqlist}{FP}
		\req{010}{Abbruch des Beweisvorgangs}{Der Benutzer soll in der Lage sein, einen gestarteten Beweisvorgang abzubrechen.}
        \req{020}{Vor- und Nachbedingungen}{Vorbedingungen für Funktionen werden als Annahmen behandelt, die in der gesamten Funktion gelten, und Nachbedingungen als Zusicherungen, die bei Beendigung der Funktion erfüllt sein müssen.}%
        \req{030}{Invarianten}{Invarianten für Schleifen werden als Zusicherungen behandelt, die nach jedem Schleifendurchlauf erfüllt sein müssen.}%
	\end{reqlist}

\subsection{Grafische Benutzeroberfläche (GUI)}
% TODO Nummerierung
\begin{reqlist}{FG}
		\req{060}{Breakpoints}{Der Benutzer kann Breakpoints direkt im Editor setzen und löschen.}
		\req{070}{Breakpoint-""\"{U}bersicht}{Die GUI listet in einem separaten Fenster alle gesetzten Breakpoints auf. Der Benutzer kann direkt in diesem Fenster Breakpoints editieren und löschen.}
		\req{080}{Fehler\"{u}bersicht}{In einem separaten Fenster wird eine \"{U}bersicht aller Parserfehler mit einer Fehlerbeschreibung angezeigt. Aus dieser Liste kann der Benutzer den Editor an der Fehlerposition aufrufen.}
		\req{090}{Programmausf"uhrung}{Der Benutzer kann den Interpreter aus der GUI heraus starten, wobei die Wahl zwischen dem Debug- und dem Ausf\"{u}hrungsmodus besteht.}
		\req{130}{Beweiseraufruf}{Es existiert ein GUI-""Befehl, um die Programmspezifikation vom Beweiser überprüfen zu lassen.}
% TODO Musskriterien
		\req{010}{Hilfe}{Der Benutzer kann die Dokumentation zur Programmiersprache aus der Entwicklungsumgebung heraus über einen Befehl aufrufen.}
	    \req{010}{Aufruf der Beispielsammlung}{Der Benutzer kann eine Liste der mitgelieferten Beispielprogramme aufrufen. Nach Auswahl eines Beispielprogramms lädt die Entwicklungsumgebung dieses Programm.}
\end{reqlist}
% TODO Wunschkriterien
\begin{reqlist}{FG}
		\req{020}{Kontextsensitive Hilfe}{Der Benutzer kann die Dokumentation zu einem gerade ausgewählten Element der Programmier- oder Annotationssprache über einen Befehl aufrufen.}
		\req{030}{Laden von Beispielprogrammen aus der Dokumentation heraus}{Der Benutzer kann ein in der Hilfe gezeigtes Beispielprogramm direkt aus dem Hilfefenster in den Editor laden, um es dort auszuführen.}
\end{reqlist}

\subsubsection{Editor-""Funktionen}

Die GUI stellt einen Editor f\"{u}r die spezifizierte \see{WHILE-""Sprache} zur Verf\"{u}gung. Folgende Funktionen unterst\"{u}tzen den Benutzer bei der Eingabe des Programmcodes:

% TODO Musskriterien
	\begin{reqlist}{FG}
		\req{140}{Ergebnisse des Beweisers}{Nach Beendigung des Beweisers gibt das Programm das entsprechende Ergebnis auf dem Bildschirm aus. Die mögliche Angabe eines Gegenbeispiels wird übersetzt und die entsprechende Annotation im Code kenntlich gemacht. Eine mögliche Fehlerstelle wird ebenfalls im Code markiert.}
	\end{reqlist}
% TODO Sollkriterien
	\begin{reqlist}{FG}
		\req{050}{Unicode-""Unterst\"{u}tzung}{Die Verwendung von Unicode-""Symbolen f"{u}r logische Operatoren und Quantoren wird unterst"{u}tzt. Dabei soll die Eingabe solcher Symbole durch die grafische Benutzeroberfl"ache erleichtert werden.}
	\end{reqlist}
% TODO Wunschkriterien
	\begin{reqlist}{FG}
		\req{010}{Syntaxhervorhebung}{Im eingegebenen Programmcode werden Elemente wie Schl\"{u}sselw\"{o}rter, Zahlen und Wahrheitswerte farblich hervorgehoben.}
		\req{020}{Fehlerkorrektur bei der Eingabe}{Parserfehler werden direkt nach der Eingabe markiert.}
		\req{030}{Codevervollst\"{a}ndigung}{Der Benutzer hat die M\"{o}glichkeit, sich an der Cursorposition eine Liste von verf\"{u}gbaren Syntaxelementen anzeigen zu lassen.}
	\end{reqlist}

\subsubsection{Debugger-""Funktionen}

% TODO Musskriterien
	\begin{reqlist}{FG}
		\req{100}{Codemarkierung}{Bei pausierter Programmausf"uhrung wird die Codezeile mit der n\"{a}chsten Anweisung markiert.}
		\req{110}{Ausdrucksauswertung}{Der Benutzer kann eine Liste von Ausdr\"{u}cken angeben, die bei jeder Pausierung einer Programmausf"uhrung ausgewertet werden und deren Auswertung in einem Teilbereich der GUI angezeigt werden.}
		\req{120}{Variablenansicht}{In einem separaten Fenster werden bei pausierter Programmausf"uhrung die Werte aller Variablen im aktuellen G\"{u}ltigkeitsbereich angezeigt. Der Benutzer kann die Werte dieser Variablen \"{a}ndern. Nach \"{A}nderung einer Variablen werden zugeh\"{o}rige Fenster (z.~B.\ Ausdrucksauswertung) aktualisiert.}
	\end{reqlist}
