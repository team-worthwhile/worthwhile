\section{Funktionale Anforderungen}
\subsection{GUI}

\subsubsection{Editor-Funktionen}

Die GUI stellt einen Editor f\"{u}r die spezifizierte \see WHILE-Sprache zur Verf\"{u}gung. Folgende Funktionen unterst\"{u}tzen den Benutzer bei der Eingabe des Programmcodes:

\begin{reqlist}{FG}

    \req{010}{Syntaxhervorhebung}{Im eingegebenen Programmcode werden Elemente wie Schl\"{u}sselw\"{o}rter, Zahlen und Wahrheitswerte farblich hervorgehoben.}
    \req{020}{Fehlerkorrektur bei der Eingabe}{Syntaxfehler werden direkt nach der Eingabe markiert.}
    \req{030}{Codevervollst\"{a}ndigung}{Der Benutzer hat die M\"{o}glichkeit, sich an der Cursorposition eine Liste von verf\"{u}gbaren Syntaxelementen anzeigen zu lassen.}
    \req{040}{\textbf{TODO:} Quick Fixes}{Liste mit Korrekturvorschl\"{a}gen f\"{u}r Syntaxfehler und -warnungen unter dem Cursor}
    \req{050}{Unicode-Unterst\"{u}tzung}{Die Verwendung von Unicode-Symbolen f"{u}r logische Operatoren und Quantoren wird unterst"{u}tzt. Dabei soll die Eingabe solcher Symbole durch die grafische Benutzeroberfl"ache erleichtert werden.}
    \req{060}{Breakpoints}{Setzen und L\"{o}schen von Breakpoints durch Klick auf die Randspalte im Editor}
    \req{070}{Breakpoint-\"{U}bersicht}{Auflistung aller gesetzten Breakpoints in einem separaten Fenster mit der M\"{o}glichkeit, diese dort zu l\"{o}schen.}
    \req{080}{Fehler\"{u}bersicht}{In einem separaten Fenster wird eine \"{U}bersicht aller Syntaxfehler mit einer Fehlerbeschreibung angezeigt. Aus dieser Liste kann der Benutzer den Editor an der Fehlerposition aufrufen.}
    \req{090}{Programmausf"uhrung}{Der Benutzer kann den Interpreter aus der GUI heraus starten, wobei die Wahl zwischen dem Debug- und dem Ausf\"{u}hrungsmodus besteht.}

\end{reqlist}


\subsubsection{Debugger-Funktionen}


\begin{reqlist}{FG}

    \req{100}{Codemarkierung}{Bei pausierter Programmausf"uhrung wird die Codezeile mit der n\"{a}chsten Anweisung markiert.}
    \req{110}{Ausdrucksauswertung}{Der Benutzer kann eine Liste von Ausdr\"{u}cken angeben, die bei jeder Pausierung einer Programmausf"uhrung ausgewertet werden und deren Auswertung in einem Teilbereich der GUI angezeigt werden.}
    \req{120}{Variablenansicht}{In einem separaten Fenster werden bei pausierter Programmausf"uhrung die Werte aller Variablen im aktuellen G\"{u}ltigkeitsbereich angezeigt. Der Benutzer kann die Werte dieser Variablen \"{a}ndern. Nach \"{A}nderung einer Variablen werden zugeh\"{o}rige Fenster (z.~B.\ Ausdrucksauswertung) aktualisiert.}


\end{reqlist}

\subsection{Interpreter/Debugger}

Der \see Interpreter kann ein \see Programm in einem von zwei verschiedenen Modi ausf"{u}hren:

\begin{itemize}

  \item
     \begin{reqlist}{FI}
        \req{010}{Debug-Modus}{Hier wird das Programm schrittweise ausgef\"{u}hrt und bei Erreichen eines Breakpoints die Programmausf\"{u}hrung pausiert.}
     \end{reqlist}


  \item
    \begin{reqlist}{FI}
        \req{020}{Ausf\"{u}hrungsmodus}{Hier wird das Programm ausgef\"{u}hrt, ohne dass der Benutzer die M\"{o}glichkeit zur Pausierung der Programmausf"uhrung hat.}
    \end{reqlist}

\end{itemize}

\begin{reqlist}{FI}

    \req{030}{Abbruch einer Programmausf"uhrung}{Ein laufendes oder pausiertes Programm kann jederzeit abgebrochen werden, womit der Interpreter in den Zustand vor Ausf\"{u}hrung des Programms zur"uckversetzt wird.}

\end{reqlist}%

Ist eine Programmausf"uhrung nach Erreichen eines Breakpoints im
Debug-Modus unterbrochen, sind folgende Debugger-Funktionen
verf"ugbar.%

\begin{reqlist}{FI}%

    \req{040}{Einzelschritt}{Die n\"{a}chste Anweisung wird ausgef\"{u}hrt und anschlie"send wird die Ausf\"{u}hrung des Programms pausiert. Wenn die ausgef"uhrte Anweisung ein Methodenaufruf war, wird im n"achsten Schritt mit der ersten Anweisung der aufgerufenen Methode fortgefahren.}

    \req{050}{\"{U}berspringen}{Die n\"{a}chste Anweisung wird ausgef\"{u}hrt und anschlie"send wird die Ausf\"{u}hrung des Programms pausiert. Wenn die ausgef"uhrte Anweisung ein Methodenaufruf war, wird die aufgerufene Methode entweder bis zum Erreichen eines Breakpoints oder einschlie"slich des R"ucksprungs ausgef"uhrt.}

    \req{060}{Ausdrucksauswertung}{Der Interpreter kann bei pausierter Programmausf"uhrung einen Ausdruck interpretieren und auswerten, der im aktuellen Zustand des Programms g\"{u}ltig ist. Syntaxfehler in einem Ausdruck werden erkannt und ausgegeben.}

\end{reqlist}

\subsection{Dokumentation zur Programmier- und Annotationssprache}%

\begin{reqlist}{FD}

\req{010}{Hilfe}
{Der Benutzer kann die Dokumentation zur Programmiersprache aus der Entwicklungsumgebung heraus über einen Befehl aufrufen.}
\req{020}{Kontextsensitive Hilfe }
{Der Benutzer kann die Dokumentation zu einem gerade ausgewählten Element der Programmier- oder Annotationssprache über einen Befehl aufrufen.}
\req{030}{Laden von Beispielprogrammen aus der Dokumentation heraus }
{Der Benutzer kann ein in der Hilfe gezeigtes Beispielprogramm direkt aus dem Hilfefenster in den Editor laden, um es dort auszuführen.}

\end{reqlist}

\subsection{Beispielprogramme}%

\begin{reqlist}{FB}

\req{010}{Aufruf der Beispielsammlung }
{Der Benutzer kann eine Liste der mitgelieferten Beispielprogramme aufrufen. Nach Auswahl eines Beispielprogramms lädt die Entwicklungsumgebung dieses Programm.}

\end{reqlist}
