
\section{Funktionale Anforderungen}
\subsection{GUI}

\subsubsection{Editor-Funktionen}

Die GUI stellt einen Editor f\"{u}r die spezifizierte \see WHILE-Sprache zur Verf\"{u}gung. Folgende Funktionen unterst\"{u}tzen den Benutzer bei der Eingabe des Programmcodes:

\begin{reqlist}{FG}

    \req{010}{Syntaxhervorhebung}{Im eingegebenen Programmcode werden Elemente wie Schl\"{u}sselw\"{o}rter, Zahlen und Zeichenketten farblich hervorgehoben.}
    \req{020}{Fehlerkorrektur bei der Eingabe}{Syntaxfehler werden direkt nach der Eingabe markiert.}
    \req{030}{Codevervollst\"{a}ndigung }{Der Benutzer hat die M\"{o}glichkeit, sich an der Cursorposition eine Liste von verf\"{u}gbaren Syntaxelementen anzeigen zulassen.}
    \req{040}{Quick Fixes(Noch zukl\"{a}ren ob enthalten)}{Liste mit Korrekturvorschl\"{a}gen f\"{u}r Syntaxfehler/-warnungen unter dem Cursor.}
    \req{050}{Unicode-Unterst\"{u}tzung }{Die Verwendung von Unicode-Symbolen f"{u}r logische Operatoren wird unterst"{u}tzt. Dabei soll die Eingabe solcher Symbole durch das grafische Benutzerinterface erleichtert werden.}
    \req{060}{Breakpoints}{Setzen, Aktivieren/Deaktivieren und L\"{o}schen von Breakpoints durch Klick auf die Randspalte im Editor}
    \req{070}{Breakpoint-\"{U}bersicht}{\"{U}bersicht aller Breakpoints in separatem Fenster mit der M\"{o}glichkeit zum Aktivieren, Deaktivieren und L\"{o}schen.}
    \req{080}{Fehler\"{u}bersicht}{In einem separaten Fenster wird eine \"{U}bersicht aller Syntaxfehler mit einer Fehlerbeschreibung angezeigt. Aus dieser Liste kann der Benutzer den Editor an der Fehlerposition aufrufen.}
    \req{090}{Ausf\"{u}hren des Programms }{Der Benutzer kann den Interpreter aus der GUI heraus starten, wobei die Wahl zwischen dem Debug- und dem Ausf\"{u}hrungsmodus besteht.}

\end{reqlist}


\subsubsection{Debug-Funktionen}


\begin{reqlist}{FG}

    \req{100}{Codemakierung}{Bei pausiertem Programm wird die Codezeile mit der n\"{a}chsten Anweisung markiert.}
    \req{110}{Ausdrucksauswertung}{Benutzer kann eine Liste von Ausdr\"{u}cken angeben, die bei jeder Pausierung des Programmablaufs ausgewertet werden und deren Ergebnisse in einem Teilbereich des Programmfensters angezeigt werden. }
    \req{120}{Variablenansicht}{In einem separaten Fenster werden bei pausiertem Programmablauf die Werte aller Variablen im aktuellen G\"{u}ltigkeitsbereich angezeigt. Der Benutzer kann die Werte dieser Variablen \"{a}ndern. Nach \"{A}nderung einer Variablen werden zugeh\"{o}rige Fenster (z.B. Ausdrucksauswertung) aktualisiert.}


\end{reqlist}

\subsection{Interpreter/Debugger}

Der \see Interpreter kann ein \see Programm in einem von zwei verschiedenen Modi ausf"{u}hren:

\begin{itemize}

  \item
     \begin{reqlist}{FI}
        \req{010}{Debugmodus}{Hier wird das Programm schrittweise ausgef\"{u}hrt, bei Erreichen eines Breakpoints wird die Programmausf\"{u}hrung pausiert.}
     \end{reqlist}


  \item
    \begin{reqlist}{FI}
        \req{020}{Ausf\"{u}hrungsmodus}{Hier wird das Programm ausgef\"{u}hrt, ohne dass der Benutzer die M\"{o}glichkeit zum Pausieren des Programmablaufs hat.}
    \end{reqlist}

\end{itemize}

\begin{reqlist}{FI}

    \req{030}{Abbruch des Programmablaufs }{Ein laufendes oder pausiertes Programm kann jederzeit abgebrochen werden, womit der Interpreter in den Zustand vor Ausf\"{u}hrung des Programms versetzt wird.Pausieren bei Erreichen eines Breakpoints im Debug-Modus}

    \req{040}{Einzelschritt}{Die n\"{a}chste Anweisung wird ausgef\"{u}hrt, danach wird die Ausf\"{u}hrung des Programms pausiert. Wenn die aktuelle Anweisung ein Methodenaufruf ist, wird die erste Anweisung der aufgerufenen Methode ausgef\"{u}hrt, ansonsten die aktuelle Anweisung.}

    \req{050}{\"{U}berspringen}{Die n\"{a}chste Anweisung wird ausgef\"{u}hrt, danach wird die Ausf\"{u}hrung des Programms pausiert. Wenn die aktuelle Anweisung ein Methodenaufruf ist, wird die Methode komplett durchlaufen, bis das Programm wieder pausiert (es sei denn, es werden unterwegs Breakpoints angetroffen)}

    \req{060}{Auswertung von Ausdr\"{u}cken}{Der Interpreter kann bei pausiertem Programm einen Ausdruck interpretieren und auswerten, der im aktuellen Zustand des Programms g\"{u}ltig ist. Syntaxfehler in einem Ausdruck werden erkannt.}

\end{reqlist}
