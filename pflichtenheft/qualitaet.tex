\section{Qualitätsanforderungen}%

\begin{reqlist}{Q}%
    \req{01}{Abbruch durch den Interpreter}{Der Interpreter darf einen Programmdurchlauf nicht von sich aus abbrechen, ohne eine ausführliche Fehlermeldung anzuzeigen.}
    \req{02}{Dokumentationsgrad}{Zu jedem Schlüsselwort oder Syntaxelement der WHILE-""Sprache existiert eine eigene Dokumentationsseite, auf der die Funktionsweise erklärt und anhand mindestens eines Beispielprogramms veranschaulicht wird.}
	\req{03}{Korrektheit des Analysewerkzeugs (I)}{Ein vom Analysewerkzeug als korrekt eingestuftes Programm muss auf jeden Fall korrekt sein.}
    \req{04}{Korrektheit des Analysewerkzeugs (II)}{Ein vom Analysewerkzeug als inkorrekt eingestuftes Programm muss auf jeden Fall inkorrekt sein.}
    \req{05}{Korrektheit des Analysewerkzeugs (III)}{Falls ein Programm weder als korrekt noch als inkorrekt vom Analysewerkzeug eingestuft werden kann, muss das Werkzeug eine entsprechende Fehlermeldung ausgeben. Das Programm darf in diesem Fall jedoch keinesfalls als inkorrekt eingestuft werden.}
    \req{06}{Beweiserabbruch}{Dass gerade der Beweiser ausgeführt wird, muss dem Benutzer angezeigt werden und der Benutzer muss die Möglichkeit haben, die Ausführung des Beweisers abzubrechen. Dabei geht es insbesondere um die Behandlung von Fällen, in denen der Beweiser unbestimmt lange läuft und unbekannt ist, ob er zu einer Entscheidung kommt.}%
    \req{07}{Gegenbeispielanzeige}{Im Idealfall müssen Modelle, die der Beweiser bei der Nichterfülltheit einer Spezifikation liefert, im Quelltext auf den fehlerhaften Codeabschnitt zurückgeführt werden. Unter allen Umständen muss derjenige Teil der Spezifikation~(eine Formel in den Annotationen) markiert werden, für welche der Beweiser ein Gegenbeispiel gefunden hat, und das zurückgelieferte Modell muss in der WHILE-""Sprache (Belegungen der deklarierten Variablen bzw. Parameter, Funktionsdefinitionen) angezeigt werden.}%
    \req{08}{Unabhängigkeit von Programmdurchläufen}{Mehrere Programmdurchläufe, die gleichzeitig oder nacheinander ablaufen, beeinflussen einander nicht. Mehrere verschiedene Durchläufe ein und desselben Programms erzeugen bei gleicher Eingabe jederzeit dieselbe Ausgabe.}
\end{reqlist}
