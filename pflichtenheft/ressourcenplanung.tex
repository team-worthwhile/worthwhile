\section{Zeit- und Ressourcenplanung}%

\subsection{Implementierungsaufwand}%

\begin{figure}[H]
  \begin{tabular}{| l | l | l | }
    \hline
    \textbf{Modul} & \textbf{Aufwand (h)} & \textbf{Personen} \\ \hline
    \see{Parser} & 25 & a \\ \hline
    \see{Interpreter} & 60 & b \\ \hline
    \see{Run-time-checker} & 60 & c \\ \hline
    Schnittstelle zum \see{Beweiser} & 90 & d, e \\ \hline
    Grafische Benutzeroberfläche & 70 & e, f \\ \hline
    Dokumentation & 15 & a \\ \hline
    Beispielsammlung & 15 & a \\ \hline
    Unit- und GUI-Tests & 15 & f \\ \hline
  \end{tabular}
\end{figure}

\subsection{Gesamtaufwand}%

Die gesamte zur Verfügung stehende Arbeitszeit beträgt geschätzte 290~Arbeitsstunden pro Person, insgesamt also 1740~Personenarbeitsstunden.

Die Arbeitszeit pro Person berechnet sich wie folgt: $30\frac{h}{LP} \cdot 8LP \cdot 1,2$~(aufgerundet).

\subsection{Phasenverantwortliche}%

\begin{tabular}{| l | l | }
    \hline
    \textbf{Phase} & \textbf{Verantwortlicher} \\ \hline
    Pflichtenheft & Chris Hiatt \\ \hline
    Entwurf & Joachim Priesner \\ \hline
    Implementierung & Stefan Orf, Matthias Wagner \\ \hline
    Qualitätssicherung & Leon Handreke \\ \hline
    Abnahme/Abschlusspräsentation & Fabian Ruch \\ \hline
\end{tabular}

\subsection{Ressourcen}%

\begin{itemize}%
    \item Rechner, der die Mindestanforderungen erfüllt
\end{itemize}%
