\section{Zeit- und Ressourcenplanung}%

\subsection{Implementierungsaufwand}%

\begin{figure}[H]
  \begin{tabular}{| l | l | l | }
    \hline
    \textbf{Modul} & \textbf{ca. Aufwand (h)}\\ \hline
    \see{Parser} & 25\\ \hline
    \see{Interpreter} & 70 \\ \hline
    \see{Run-time-""Checker} & 70\\ \hline
    Schnittstelle zum \see{Beweiser} & 90 \\ \hline
    Grafische Benutzeroberfläche & 40 \\ \hline
    Dokumentation & 20 \\ \hline
    Beispielsammlung & 15\\ \hline
    Unit- und GUI-""Tests & 20 \\ \hline \hline
    \textbf{Gesamt} & 350 \\ \hline
  \end{tabular}
\end{figure}

\subsection{Gesamtaufwand}%

Die gesamte zur Verfügung stehende Arbeitszeit beträgt geschätzte 290~Arbeitsstunden pro Person, insgesamt also 1740~Personenarbeitsstunden.

Die Arbeitszeit pro Person berechnet sich wie folgt aus den Leistungspunkten~(LP) für die PSE-""Veranstaltung: $30\textrm{h}/\textrm{LP} \cdot 8~\textrm{LP} \cdot 1,2$~(aufgerundet).

\subsection{Phasenverantwortliche}%

\begin{tabular}{| l | l | }
    \hline
    \textbf{Phase} & \textbf{Verantwortlicher} \\ \hline
    Pflichtenheft & Chris Hiatt \\ \hline
    Entwurf & Fabian Ruch \\ \hline
    Implementierung & Stefan Orf, Matthias Wagner \\ \hline
    Qualitätssicherung & Leon Handreke \\ \hline
    Abnahme/Abschlusspräsentation & Joachim Priesner\\ \hline
\end{tabular}

