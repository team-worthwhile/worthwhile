\section{Zeit- und Ressourcenplanung}%

\subsection{Zeitaufwand}%

%\begin{left}
  \begin{tabular}{| l | l | l | }
    \hline
    Modul & Zeitaufwand & Personen \\ \hline
    Parser &  &  \\ \hline
    Interpreter &  &  \\ \hline
    Laufzeitprüfer &  &  \\ \hline
    Schnittstelle zum Beweiser &  &  \\ \hline
    Grafische Benutzeroberflaeche &  &  \\ \hline
    Dokumentation &  &  \\ \hline
    Beispielsammlung &  &  \\ \hline
    Testen &  &  \\ \hline
    \hline
  \end{tabular}
%\end{center}

Die gesamte zur Verfuegung stehende Arbeitszeit betraegt geschaetzte 290 Arbeitsstunden pro Person, insgesamt also 1740 Personenarbeitsstunden.
Die Arbeitszeit pro Person berechnet sich wie folgt: 30h/LP * 8LP * 1,2 (aufgerundet)

\subsection{Phasenverantwortliche}%

%\begin{left}
  \begin{tabular}{| l | l | }
    \hline
    Phase & Verantwortlicher \\ \hline
    Pflichtenheft & Chris Hiatt \\ \hline
    Entwurf & ? \\ \hline
    Implementierung & ?,? \\ \hline
    Qualitätssicherung & ? \\ \hline
    Abnahme / Abschlusspräsentation & ? \\ \hline
    \hline
  \end{tabular}
%\end{center}

\subsection{Ressourcen}%

\begin{itemize}%
    \item Rechner,der die Mindestanforderungen erfüllt
\end{itemize}%
