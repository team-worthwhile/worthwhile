\section{Globale Testfälle und Testszenarien}

\begin{reqlist}{T}
	\req{01}{Randfälle korrekter Programme}{
		\testspec
		{Korrektes Verhalten des Parsers, Interpreters und Beweisers bei Programmen, die der Sprachspezifikation entsprechen, aber gewisse Randfälle darstellen}
		{Die Entwicklungsumgebung ist gestartet und es ist kein Programm geladen.}
		{
			\begin{enumerate}
				\item Öffnen des Testprogramms:
					\begin{reqlist}{T}
						\req{01a}{Leeres Programm}{Das leere Programm, d.~h.\ die Datei enthält keine Anweisungen (aber eventuell Leerzeilen)}
						\req{01b}{Verschachteltes Programm}{Ein Testprogramm mit verschachtelten Schleifen und bedingten Anweisungen bis zu einer Schachtelungstiefe von 1000. Die Abfragen und Schleifen sind so gestaltet, dass alle Anweisungen erreichbar sind. Das Programm enthält keine Zusicherungen.}
						\req{01c}{Langes Programm}{Ein Testprogramm ohne bedingte Anweisungen, Schleifen und Zusicherungen, aber mit mehr als 10.000 Anweisungszeilen.}
					\end{reqlist}
				\item Ausführen des Programms durch den entsprechenden Befehl
				\item Überprüfen des Programms durch den entsprechenden Befehl
			\end{enumerate}
		}
		{
			\begin{enumerate}
				\item Der Interpreter wird ohne Fehlermeldung beendet.
				\item Das Überprüfen des Programms schlägt nicht fehl (entweder wird Erfolg zurückgemeldet oder eine Warnung wegen fehlender Zusicherungen ausgegeben).
			\end{enumerate}	
		}
	}
	
	\req{02}{Fehlerhafte Programme}{
		\testspec
		{Korrektes Verhalten des Parsers, Interpreters und Beweisers bei Programmen, die nicht der Sprachspezifikation entsprechen}
		{Die Entwicklungsumgebung ist gestartet und es ist kein Programm geladen.}
		{
			\begin{enumerate}
				\item Öffnen des Testprogramms:
					\begin{reqlist}{T}
						\req{02a}{Falsch geschriebene/nicht existierende Schlüsselwörter}{Es werden in einem korrekten Programm Schlüsselwörter gegen andere Wörter, die keine Schlüsselwörter sind, ausgetauscht.}
						\req{02b}{Fehlender Leerraum}{Es werden in einem korrekten Programm Leerzeichen so gelöscht, dass das resultierende Programm nicht mehr der Sprachspezifikation entspricht.}
						\req{02c}{Fehlende/überflüssige Operanden}{Es wird eine binäre mathematische oder logische Operation mit nur einem Operanden oder eine unäre mathematische oder logische Operation mit zwei Operanden aufgerufen.}
						\req{02d}{Operanden vom falschen Typ}{Es wird eine mathematische oder logische Operation mit Operanden aufgerufen, die nicht dem von der Operation geforderten Typ entsprechen. \\ \emph{Dieser Testfall findet nur Anwendung, wenn die Sprachspezifikation eine automatische Typkonvertierung von Variablen nicht erlaubt.}}
						\req{02e}{Ungültige Parameter bei Funktionsaufruf}{In einem Funktionsaufruf werden Parameter übergeben, die in Anzahl und/oder Typ nicht der Funktionssignatur entsprechen. \\ \emph{Dieser Testfall findet nur Anwendung, wenn die Sprachspezifikation eine automatische Typkonvertierung von Variablen nicht erlaubt.}}	
						\req{02f}{Fehlerhafter Funktionsaufruf}{Es wird versucht, eine nicht deklarierte Funktion aufzurufen.}
						\req{02g}{Verwenden nicht deklarierter Variablen}{Es wird versucht, eine Variable zu verwenden, die nicht deklariert wurde. \\ \emph{Dieser Testfall findet nur Anwendung, wenn die Sprachspezifikation das Deklarieren der Variablen vor ihrer Verwendung vorschreibt.}}
						\req{02h}{Unerlaubte Typkonvertierung}{Es wird versucht, einer Variable einen Wert zuzuweisen, der nicht ihrem Typ entspricht. \\ \emph{Dieser Testfall findet nur Anwendung, wenn die Sprachspezifikation eine automatische Typkonvertierung von Variablen nicht erlaubt.}}
					\end{reqlist}
				\item Ausführen des Programms durch den entsprechenden Befehl
				\item Überprüfen des Programms durch den entsprechenden Befehl
			\end{enumerate}
		}
		{
			\begin{enumerate}
				\item Die fehlerhafte Stelle wird im Editorfenster markiert.
				\item In der Fehlerliste wird eine Fehlerbeschreibung angezeigt.
				\item Durch Doppelklick auf den Eintrag in der Fehlerliste springt der Cursor an die fehlerhafte Stelle im Programm.
				\item Beim Versuch, das Programm auszuführen, wird eine (allgemeine oder spezifische) Fehlermeldung ausgegeben.
				\item Beim Versuch, das Programm zu überprüfen, wird eine (allgemeine oder spezifische) Fehlermeldung ausgegeben.
			\end{enumerate}	
		}
	}
	
	\req{03}{Syntaxhervorhebung im Editor}{
		\testspec{Korrekte Syntaxhervorhebung im Editor}
		{keine}
		{\begin{enumerate}
			\item Es wird ein Programm geladen, das alle definierten Sprachkonstrukte mindestens einmal enthält.
		\end{enumerate}}
		{
		\begin{enumerate}
			\item Es werden alle in der Sprache definierten Elemente~(Schlüsselwörter, Funktions- und Variablennamen, Kommentare, Zahlen) gemäß den Einstellungen in der Entwicklungsumgebung hervorgehoben.
		\end{enumerate}
		}
	}
	
	\req{04}{Debuggen eines Programms}{
		\testspec{Korrektes Verhalten des Interpreters bei Ausführung der Debug-""Operationen}
		{Es ist ein Programm geladen, das der Sprachspezifikation entspricht und alle definierten Sprachkonstrukte mindestens einmal enthält.}
		{\begin{enumerate}
			\item Mittels des entsprechenden Befehls in der Entwicklungsumgebung werden ein oder mehrere Haltepunkte gesetzt.
			\item Der Interpreter wird über den entsprechenden Befehl im Debug-""Modus gestartet.
			\item Nach Anhalten des Programms an einem Haltepunkt werden mehrmals hintereinander der Einzelschritt-""Befehl und der Überspringen-""Befehl ausgeführt. Dabei sollen beide Befehle auch in Zeilen ausgeführt werden, in denen sich Funktionsaufrufe befinden.
			\item Es wird einer der folgenden Schritte ausgeführt (der Testfall muss zweimal durchgeführt werden, damit beide Schritte abgedeckt werden):
			\begin{enumerate}
			\item Der Programmablauf wird durch den entsprechenden Befehl abgebrochen.
			\item Der Programmablauf wird durch den entsprechenden Befehl fortgesetzt.			
			\end{enumerate}
			\item Wie oben, allerdings wird der Interpreter im Ausführungsmodus gestartet.
		\end{enumerate}}
		{\begin{enumerate}
			\item Die Haltepunkte werden im Editor und in der Haltepunktübersicht angezeigt.
			\item Nach Ausführen des Programms im Debug-""Modus hält der Interpreter vor der Zeile, die den ersten Haltepunkt enthält.
			\item Bei Ausführung des Einzelschritt-""Befehls führt der Interpreter den nächsten Befehl aus, wobei er in Funktionsaufrufe springt.
			\item Bei Ausführung des Überspringen-""Befehls führt der Interpreter den nächsten Befehl aus, wobei er nicht in Funktionsaufrufe springt.
			\item Bei Ausführung des Abbruchbefehls wird der Programmablauf sofort abgebrochen.
			\item Bei Ausführung des Fortsetzen-""Befehls führt der Interpreter das Programm bis zum nächsten Haltepunkt bzw.\ bis zum Programmende aus.
			\item Nach Ausführen des Programms im Ausführungsmodus ignoriert der Interpreter die Haltepunkte und führt das Programm bis zum Ende aus.
			\item Bei Ausführung im Debug- und Ausführungsmodus verhält sich das Programm gleich (beispielsweise in Bezug auf Terminierung des Programmablaufs).
		\end{enumerate}}
	}
	
	\req{05}{Ansprechbarkeit der Benutzeroberfläche}{
		\testspec{Ansprechbarkeit der Benutzeroberfläche bei Ausführung eines Programms}
		{Es ist ein Programm geladen, das der Sprachspezifikation entspricht und eine Endlosschleife enthält.}
		{\begin{enumerate}
			\item Der Benutzer startet den Interpreter.	
			\item Während der Interpreter läuft, bricht der Benutzer das Programm durch den entsprechenden Befehl ab.
		\end{enumerate}}
		{\begin{enumerate}
			\item Der Interpreter reagiert in weniger als einer Sekunde auf den Befehl und bricht die Ausführung des Programms ab.
		\end{enumerate}}
	}
	
	\req{06}{Ändern von Variablenwerten während der Ausführung eines Programms}{
		\testspec{Korrekte Auswirkung der Änderung von Variablenwerten}
		{Es ist ein Programm geladen und mehrere Haltepunkte sind im Programm gesetzt.}
		{\begin{enumerate}
			\item Der Benutzer startet den Interpreter im Debug-""Modus.	
			\item Nach Halt des Interpreters an einem der Haltepunkte gibt der Benutzer im Fenster zur Ausdrucksauswertung einige Ausdrücke ein, darunter sowohl gültige als auch ungültige (Syntaxfehler, nicht deklarierte Variablen etc.).
			\item Der Benutzer ändert im Variablenfenster die Werte von Variablen so, dass der Programmablauf dadurch beeinflusst wird (zum Beispiel Abbruch einer Schleife).
			\item Der Benutzer setzt die Programmausführung fort.
		\end{enumerate}}
		{\begin{enumerate}
			\item Die Ausdrücke werden korrekt ausgewertet und für nicht auswertbare Ausdrücke wird eine aussagekräftige Fehlermeldung angezeigt.
			\item Nach Änderung der Variablenwerte werden die Werte der Ausdrücke im Ausdrucksfenster aktualisiert.
			\item Beim Fortsetzen der Programmausführung werden die neuen Variablenwerte übernommen und das Programm entsprechend ausgeführt.
		\end{enumerate}}
	}
	
	\req{07}{Korrekte Speicherung der Optionen}{
		\testspec{Korrektes Speichern und Laden der Optionen in der GUI}
		{Die GUI ist geöffnet.}
		{\begin{enumerate}
			\item Der Benutzer öffnet eine Datei.
			\item Der Benutzer setzt einen Haltepunkt in der Datei.
			\item Der Benutzer öffnet den Einstellungsdialog und verändert alle Einstellungen so, dass sie nicht den Standardeinstellungen entsprechen.
			\item Der Benutzer beendet die GUI und startet sie erneut.
			\item Der Benutzer öffnet den Einstellungsdialog.
		\end{enumerate}}
		{\begin{enumerate}
			\item Nach dem erneuten Starten der GUI werden die beim Beenden geöffneten Dateien erneut geladen.
			\item Der gesetzte Haltepunkt ist vorhanden.
			\item Die Einstellungen im Einstellungsdialog entsprechen den vorher getroffenen Einstellungen, weichen also insbesondere von den Standardeinstellungen ab.
		\end{enumerate}}
	}
	
	\req{08}{Testen des Überprüfers}{
		\testspec
		{Korrektes Funktionieren der Schnittstelle zum Überprüfer}
		{Die Entwicklungsumgebung ist gestartet und es ist kein Programm geladen.}
		{
			\begin{enumerate}
				\item Öffnen des Testprogramms:
					\begin{reqlist}{T}
						\req{08a}{Testprogramm für Annotationen}{Ein Programm, welches je mindestens eine Annotation jedes Annotationstyps enthält.}
						\req{08b}{Programm mit verschachtelten Schleifen}{Ein Testprogramm mit mehreren ineinander verschachtelten \texttt{while}-""Schleifen.}
						\req{08c}{Rekursives Programm}{Ein Programm, das entweder eine Funktion enthält, die sich selbst aufruft, oder mehrere Funktionen, die einander zyklisch aufrufen.}
						\req{08d}{Funktionsaufrufe in Annotationen}{Ein Programm, das eine Annotation enthält, die einen Funktionsaufruf enthält.}
					\end{reqlist}
				\item Überprüfen des Programms durch den entsprechenden Befehl
			\end{enumerate}
		}
		{
			\begin{enumerate}
				\item Der Überprüfer wird ordnungsgemäß ausgeführt und gibt insbesondere keine Fehlermeldung bezüglich der Programmsyntax zurück.
				\item Der Überprüfungsvorgang liefert das (für das jeweilige Testprogramm) korrekte Ergebnis zurück.
			\end{enumerate}	
		}
	}
	
\end{reqlist}
