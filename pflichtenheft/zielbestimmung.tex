\section{Zielbestimmung}%

\textbf{TODO:} Bisher nur aus der Pr"asentation: Umformulierungen vornehmen besonders bei den Punkten, die nicht Muss-Kriterien sind.%

\textbf{TODO:} Nachverfolgbarkeit der Anforderungen: Jeder funktionalen Anforderung muss eine Zielbestimmung zuzuordnen sein $\Rightarrow$ evtl. Zielbestimmungen erg"anzen%

\subsection{\see Interpreter}%

Ermöglicht die (schrittweise) Ausführung eines \see Programms und die Inspektion des aktuellen \see Programmzustands durch den Benutzer.%

\begin{description}%
    \item[Wunsch] Benutzer kann den Programmzustand w"ahrend der Ausf"uhrung "andern%
    \item[Wunsch] Auswertung von benutzerdefinierten \see Ausdr"ucken "uber Programmzustand%
\end{description}%

\subsection{\see Run-time checker}%

Erlaubt die Pr"ufung von \see Zusicherungen in Programmen eines konkreten Programmdurchlaufs zur Laufzeit.%

\begin{description}%
    \item [Muss] Auswertung der im Programm eingebetteten (quantorenfreien) \see Annotationen; R"uckmeldung im Fehlerfall "uber die grafische Benutzeroberfl"ache%
    \item [Soll] Auswertung von Formeln mit \see Quantoren "uber eingeschr"ankten Bereich%
    \item [Wunsch] Auswertung von Formeln mit Quantoren mit Hilfe eines \see Beweisers (Dabei wird der aktuelle Zustand des Programmes dem Beweiser "ubergeben.)%
\end{description}%

\subsection{Grafische Benutzeroberfl"ache}%

Dient zur Steuerung der einzelnen Komponenten des Systems sowie der Anzeige von R"uckmeldungen der (externen) \see Module des Werkzeugs.%

\begin{description}%
    \item [Muss] Sprache des Benutzeroberfläche: Englisch%
    \item [Abgrenzung] nur Optionen und Funktionen, die der besseren oder einfacheren Eingabe und Analyse des Programms durch den Benutzer zutr"aglich sind%
    \item [Wunsch] Es sollen zur besseren Lesbarkeit \see Unicode-Symbole f"ur \see logische Operatoren angezeigt werden.%
    \item [Wunsch] Verwaltung von alternativen \see Beweisverpflichtungen (\_axiom; in Funktionen: \_requires, \_ensures)%
\end{description}%

\subsection{\see WHILE-Sprache}%

\begin{description}%
    \item [Muss] Umfang der WHILE-Sprache wie vorgestellt, zus"atzlich: Arrays%
    \item [Wunsch] Methodenaufrufe%
    \item [Abgrenzung] nur $\mathbb{Z}$, Boolean und Arrays als Datentypen. Keine Strings, Gleitkommazahlen, Pointer usw.%
    \item [Abgrenzung] kein \see Heap%
    \item [Abgrenzung] keine \see Nebenl"aufigkeit%
\end{description}%

\subsection{Annotationssprache}%

\begin{description}%
    \item [Muss] Spezifikationen sind im Quelltext des Programms eingebettet, aber durch spezielle Syntax klar vom Programm getrennt.%
    \item [Muss] Syntax und Semantik von Ausdr"ucken werden aus der Programmiersprache "ubernommen (soweit m"oglich).%
    \item [Muss] Zusicherungen erlauben Aussagen der \see Pr"adikatenlogik.%
    \item [Muss] Grundlegende Annotationen: \_assert, \_assume%
    \item [Soll] syntactic sugar f"ur \see Vor-/Nachbedingungen, \see (In)varianten, \see globale Annahmen%
    \item [Wunsch] M"oglichkeit, verschiedene (getrennte) Vor-/Nachbedingungen anzugeben%
\end{description}%

\subsection{Dokumentation und Beispielsammlung}%

\begin{description}%
    \item [Muss] Dokumentation f"ur die verwendete Programmier- und Annotationssprache%
    \item [Muss] Mitgelieferte Sammlung von Beispielprogrammen%
\end{description}%
