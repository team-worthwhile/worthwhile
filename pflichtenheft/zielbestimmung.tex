\section{Zielbestimmung}%

\textbf{TODO:} Bisher nur aus der Pr"asentation: Umformulierungen vornehmen besonders bei den Punkten, die nicht Muss-Kriterien sind.%

\textbf{TODO:} Nachverfolgbarkeit der Anforderungen: Jeder funktionalen Anforderung muss eine Zielbestimmung zuzuordnen sein $\Rightarrow$ evtl. Zielbestimmungen erg"anzen%

\subsection{Interpreter}%

Der \see Interpreter ermöglicht die (schrittweise) Ausführung eines \see Programms und die Inspektion des aktuellen \see Programmzustands durch den Benutzer.%

\begin{description}%
    \item[Wunsch] Benutzer kann den Programmzustand w"ahrend der Ausf"uhrung "andern%
    \item[Wunsch] Auswertung von benutzerdefinierten \see Ausdr"ucken "uber dem Programmzustand%
\end{description}%

\subsection{Run-time checker}%

Der \see Run-time checker erlaubt die Pr"ufung von \see Zusicherungen in Programmen f"ur einen konkreten Programmdurchlauf, d.~h.\ zur Laufzeit.%

\begin{description}%
    \item [Muss] Auswertung der im Programm eingebetteten (quantorenfreien) \see Annotationen%
        \begin{itemize}%
            \item R"uckmeldung im Fehlerfall "uber die grafische Benutzeroberfl"ache%
        \end{itemize}%
    \item [Soll] Auswertung von Formeln mit \see Quantoren "uber eingeschr"ankten Bereich%
    \item [Wunsch] Auswertung von Formeln mit Quantoren mit Hilfe eines \see Beweisers%
        \begin{itemize}%
            \item Der aktuelle Programmzustand wird dazu dem Beweiser "ubergeben.%
        \end{itemize}%
\end{description}%

\subsection{Grafische Benutzeroberfl"ache}%

Die grafische Benutzeroberfl"ache dient zur Steuerung der einzelnen Komponenten des Systems sowie der Anzeige von R"uckmeldungen der (externen) \see Module des Werkzeugs.%

\begin{description}%
    \item [Muss] Sprache des Benutzeroberfläche: Englisch%
    \item [Abgrenzung] nur Optionen und Funktionen, die der besseren oder einfacheren Eingabe und Analyse des Programms durch den Benutzer zutr"aglich sind%
    \item [Wunsch] Es sollen zur besseren Lesbarkeit \see Unicode-Symbole f"ur \see logische Operatoren angezeigt werden.%
    \item [Wunsch] Verwaltung von alternativen \see Beweisverpflichtungen (\texttt{\_axiom} sowie speziell f"ur Funktionen \texttt{\_requires} und \texttt{\_ensures})%
\end{description}%

\subsection{WHILE-Sprache}%

Die vom Worthwhile-Interpreter erkannte Programmiersprache ist eine \see WHILE-Sprache.%

\begin{description}%
    \item [Muss] Grundlegende Sprachkonstrukte: \texttt{if}-Anweisungen und \texttt{while}-Schleifen%
    \item [Muss] Variablen mit Ganzzahlen (\texttt{Integer}), Wahrheitswerten (\texttt{Boolean}) sowie Arrays von \texttt{Integer} und \texttt{Boolean} als Datentypen%
    \item [Wunsch] Definition und Aufruf von eigenen Funktionen%
    \item [Abgrenzung] Weder Strings, Gleitkommazahlen, Pointer noch eigene Datentypen%
    \item [Abgrenzung] Kein \see Heap%
    \item [Abgrenzung] Keine \see Nebenl"aufigkeit%
\end{description}%

\subsection{Annotationssprache}%

In der \see Annotationssprache werden die vom Beweiser gepr"uften WHILE-Programme spezifiziert. Zu einer Spezifikation geh"oren sowohl Zusicherungen als auch Invarianten.%

\begin{description}%
    \item [Muss] Spezifikationen sind im Quelltext des Programms eingebettet, aber durch spezielle Syntax klar vom Programm getrennt.%
    \item [Muss] Syntax und Semantik von Ausdr"ucken werden aus der Programmiersprache "ubernommen (soweit m"oglich).%
    \item [Muss] Zusicherungen erlauben Aussagen der \see Pr"adikatenlogik.%
    \item [Muss] Grundlegende Annotationen: \texttt{\_assert} und \texttt{\_assume}%
    \item [Soll] syntactic sugar f"ur \see Vor-/Nachbedingungen, \see (In)varianten, \see globale Annahmen%
    \item [Wunsch] M"oglichkeit, verschiedene (getrennte) Vor-/Nachbedingungen anzugeben%
\end{description}%

\subsection{Dokumentation und Beispielsammlung}%

\begin{description}%
    \item [Muss] Dokumentation f"ur die verwendete Programmier- und Annotationssprache%
    \item [Muss] Mitgelieferte Sammlung von kommentierten und spezifizierten Beispielprogrammen zur Verdeutlichung der Funktionalit"at von \textit{Worthwhile}%
\end{description}%
