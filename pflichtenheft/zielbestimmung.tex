\section{Zielbestimmung}%

% TODO Nachverfolgbarkeit der Anforderungen: Jeder funktionalen Anforderung muss eine Zielbestimmung zuzuordnen sein $\Rightarrow$ evtl. Zielbestimmungen ergänzen

\subsection{Interpreter}%

Der \see{Interpreter} ermöglicht die (schrittweise) Ausführung eines Programms und die Inspektion des aktuellen Programmzustands durch den Benutzer.%

\begin{description}%
    \item[Wunsch] Der Benutzer kann den Programmzustand w"ahrend der Ausf"uhrung "andern.%
    \item[Wunsch] Während eines Programmdurchlaufs können benutzerdefinierte Ausdr"ucke "uber dem Programmzustand ausgewertet werden.%
\end{description}%

\subsection{Run-time-Checker}%

Der \see{Run-time-Checker} erlaubt die Pr"ufung von \see{Zusicherungen} in Programmen f"ur einen konkreten Programmdurchlauf, d.~h.\ zur Laufzeit.%

\begin{description}%
    \item [Muss] Die im Programm eingebetteten (quantorenfreien) \see{Annotationen} werden ausgewertet. Im Fehlerfall erfolgt die Rückmeldung "uber die grafische Benutzeroberfl"ache.
    \item [Soll] Eingebettete Formeln mit Quantoren "uber einem eingeschr"ankten Bereich werden ausgewertet.%
    \item [Wunsch] Eingebettete Formeln mit Quantoren über uneingeschränkten Bereichen werden mit Hilfe eines \see{Beweisers} ausgewertet. Der aktuelle Programmzustand wird dazu dem Beweiser "ubergeben.
\end{description}%

\subsection{Grafische Benutzeroberfl"ache}%

Die grafische Benutzeroberfl"ache dient zur Steuerung der einzelnen Komponenten des Systems sowie der Anzeige von R"uckmeldungen der (externen) Module des Werkzeugs.%

\begin{description}%
    \item [Muss] Die Sprache der Benutzeroberfläche ist Englisch.%
    \item [Abgrenzung] Die Benutzeroberfläche enthält nur Optionen und Funktionen, die der besseren oder einfacheren Eingabe und Analyse eines Programms durch den Benutzer zutr"aglich sind.%
    \item [Wunsch] Es werden zur besseren Lesbarkeit Unicode-Symbole f"ur logische Operatoren und Quantoren angezeigt.%
\end{description}%

\subsection{WHILE-Sprache}%

Die vom \textit{Worthwhile}-Interpreter erkannte Programmiersprache ist eine \see{WHILE-Sprache}.

\begin{description}%
    \item [Muss] Grundlegende Sprachkonstrukte: \texttt{if}-Anweisungen und \texttt{while}-Schleifen
    \item [Muss] Variablen mit Ganzzahlen~(\texttt{Integer}), Wahrheitswerten~(\texttt{Boolean}) sowie Arrays von \texttt{Integer} und \texttt{Boolean} als Datentypen%
    \item [Wunsch] Definition und Aufruf von eigenen Funktionen%
    \item [Abgrenzung] Weder Strings, Gleitkommazahlen, Zeiger noch eigene Datentypen%
    \item [Abgrenzung] Kein Heap%
    \item [Abgrenzung] Keine Nebenl"aufigkeit%
\end{description}%

\subsection{Annotationssprache}%

In der \see{Annotationssprache} werden die vom Beweiser gepr"uften WHILE-Programme spezifiziert. Zu einer Spezifikation geh"oren sowohl Zusicherungen als auch Invarianten.%

\begin{description}%
    \item [Muss] Spezifikationen sind im Quelltext des Programms eingebettet, aber durch spezielle Syntax klar vom Programm getrennt.%
    \item [Muss] Syntax und Semantik von Ausdr"ucken werden, soweit m"oglich, aus der Programmiersprache "ubernommen.%
    \item [Muss] Grundlegende Annotationen: Zusicherungen (Assertions) und Annahmen (Assumptions)%
    \item [Muss] Zusicherungen erlauben Aussagen der \see{Prädikatenlogik} "uber den Zustand einer Programmausf"uhrung.%
    \item [Soll] Spezielle Syntax~("`syntactic sugar"'), um die Verst"andlichkeit von Annotationen zu verbessern und ihre Formulierung intuitiver zu gestalten:%
        \begin{itemize}%
            \item Vor- und Nachbedingungen an Funktionen%
            \item (In-)Varianten an Schleifen%
            \item Globale Annahmen in Programmspezifikationen%
        \end{itemize}%
    \item [Wunsch] M"oglichkeit, mehrere Vor- und Nachbedingungen f"ur eine Funktion anzugeben%
\end{description}%

\subsection{Beweiserschnittstelle}%

\begin{description}%
    \item [Muss] Übersetzung des Programms in eine vom Beweiser beweisbare prädikatenlogische Formel
    \item [Muss] Aufrufen des Beweisers
    \item [Abgrenzung] Keine automatische Findung von Schleifeninvarianten
\end{description}%

\subsection{Dokumentation und Beispielsammlung}%

\begin{description}%
    \item [Muss] Dokumentation f"ur die verwendete Programmier- und Annotationssprache%
    \item [Muss] Mitgelieferte Sammlung von kommentierten und spezifizierten Beispielprogrammen zur Verdeutlichung der Funktionalit"at von \textit{Worthwhile}%
\end{description}%
