\section{Änderungen gegenüber dem Entwurf}

Während der Validierungsphase haben sich gegenüber der im Entwurf festgelegten Architektur nur kleine Änderungen ergeben. So wurden einige Interfaces verändert, um kleinere Versehen in der Entwurfsphase zu korrigieren.

Außerdem wurden einige Klassen, die in der Implementationsphase in den Paketen \texttt{prover} und \texttt{interpreter} angelegt wurden, in das gemeinsam genutzte Paket \texttt{model} verschoben. Bei den verschobenen Klassen handelt es sich um Visitor"=Implementierungen, die bestimmte Syntaxelemente in einem Abstract"=Syntax"=Tree suchen oder ersetzen, und es wurde stets darauf geachtet, keine für den Interpreter oder die Beweiserschnittstelle spezifische Funktionalität in dieses Modul zu verschieben.

Nicht zuletzt wurde zur Behebung eines Fehlers, der den aufgerufenen Beweiser nicht vorzeitig beenden lies, die Timeout"=Funktionalität der Beweiserschnittstelle aus der Fassade in die speziellen Beweiseraufrufe verschoben. Diese Änderung am Entwurf hängt stark mit der Verwaltung von Prozessen in der Java"=Virtual"=Machine und dem Betriebssystem zusammen, was für einen sauberen Entwurf vernachlässigt werden konnte.

\subsection{"`Partial Proofs"'}

Um das Finden von Fehlern im Formelgenerator zu vereinfachen sowie den Beweisprozess für den Nutzer greifbarer zu machen, wurde die Möglichkeit der Generierung von Formeln geschaffen, die nur die Spezifikationskonformität eines Teils eines Programms beweisen. Dabei haben sich für die Schnittstellen zwischen der Beweiserschnittstelle und der GUI einige Änderungen ergeben. So werden zur Rückmeldung des Fortschritts des Beweises und der Ergebnisse von "`Teilbeweisen"' ähnlich wie im Interpreter Event-Listener gemäß des Observer-Entwurfsmusters verwendet. Durch diese Schnittstelle werden der GUI auch nähere Informationen bezüglich der Bedeutung der geführten Teilbeweise übergeben, die dem Nutzer in Form von Tooltips und farbig markierten Codepassagen angezeigt werden.

Bei der Modifikation des Formelgenerators wurde jedoch darauf geachtet, die Grundsatzentscheidungen, die im Entwurf gefällt wurden, nicht zu verletzen. So gibt es in \texttt{WPStrategy} weiterhin die Möglichkeit, eine einzige Formel generieren zu lassen, die die Spezifikationskonformität des ganzen Programms impliziert. Außerdem gibt \texttt{SpecificationChecker} nach einem Beweisvorgang weiterhin einen \texttt{Validity}-Wert zurück, der sich auf die Korrektheit des gesamten übergebenen Programms bezieht.
