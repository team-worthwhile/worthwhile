\section{Kompatibilitätstests}
Im Laufe der Entwicklung wurde Worthwhile auf die Kompatibilität mit verschiedenen Betriebssystemen, Rechnerarchitekturen und Java-Versionen getestet. Dabei wurde die Lauffähigkeit auf den Linux-Distributionen openSuse, Debian squeeze, Arch Linux, Fedora 14 sowie 16 und Gentoo sowie auf Windows 7 bestätigt. Bei einigen Tests wurde die offizielle Java-Implementation von Oracle verwendet, bei anderen die freie, auf OpenJDK basierende JVM "`IcedTea"' in Version 6 und 7. Mit keiner der verwendeten virtuellen Maschinen wurden Probleme festgestellt.

Wider Erwarten war es nötig, für Systeme mit Windows und Linux sowie für amd64- und i386-Architekturen die Software mit unterschiedlichen Parametern zu bauen. Dies ist auf verschiedene plattformspezifische Komponenten in der Eclipse-Plattform zurückzuführen, die beispielsweise je nach Plattform unterschiedliche Bibliotheken zum Zeichnen der Benutzeroberfläche verwendet. Der gesamte Worthwhile-spezifische Code hat sich als plattformunabhängig erwiesen.
