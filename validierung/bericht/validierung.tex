\documentclass[11pt,a4paper]{scrartcl}
\usepackage[ngerman]{babel}
\usepackage[utf8]{inputenc}
\usepackage[T1]{fontenc}
\usepackage{graphicx}
\usepackage{float}
\usepackage{fullpage}
\usepackage{amssymb}
\usepackage{amsmath}
\usepackage[nounderscore]{syntax}
\usepackage{url}                % URLs
\usepackage{tikz}               % for the grey border on the title page
\usepackage{ziffer}				% to use commas as decimal separators in math mode
\usepackage{pdflscape}
\usepackage{multirow}
\usepackage{booktabs}
\usepackage{longtable, mdwtab}
\usepackage[absolute,overlay]{textpos}

\usepackage[a4paper,left=40mm,right=30mm,top=20mm,bottom=20mm,includeheadfoot]{geometry}
\newcommand{\changefont}[3]{\fontfamily{#1} \fontseries{#2} \fontshape{#3} \selectfont}
%\parindent 0pt
%\setlength{\parskip}{12pt}
\usepackage[raiselinks=true,
						bookmarks=true,
						bookmarksopenlevel=1,
						bookmarksopen=true,
						bookmarksnumbered=true,
						hyperindex=true,
						plainpages=false,
						pdfpagelabels=true,
						pdfborder={0 0 0.5},
						colorlinks=false,
						linkbordercolor={0 0.61 0.50},
						citebordercolor={0 0.61 0.50}]{hyperref}  %{0.57 0.74 0.57}

\author{PSE-Projekt 4 Team 2}
\title{Validierungsbericht zu Worthwhile}

\setcounter{tocdepth}{1} % make TOC fit on one page

\hyphenation{Worth-while}
\hyphenation{Pro-gramm-in-halt}

\newcommand{\reqtype}{}
\newenvironment{reqlist}[1]{\begin{description} \renewcommand{\reqtype}{#1}}{\end{description}}
\newcommand{\req}[3]{\item[\textbf{/\reqtype{}#1/}] #2 \\ #3}
\newcommand{\see}[1]{#1}
\renewcommand{\int}{\texttt{Integer}}
\newcommand{\bool}{\texttt{Boolean}}

\newcommand{\method}[1]{
			\item[\texttt{\textbf{#1}}]
            }
\newcommand{\attr}[1]{\item[\texttt{\textbf{#1}}]}
\newcommand{\type}[1]{\texttt{#1}}
\newcommand{\literal}[1]{\item[\texttt{#1}]}
\newcommand{\mlmethod}[1]{\method{\parbox{\textwidth}{#1}}}
\newenvironment{rcases}{%
  \left.\renewcommand*\lbrace.%
  \begin{cases}}%
{\end{cases}\right\rbrace}
\newcommand{\guitest}[5]{\subsubsection{#1}\begin{description}
	\item[Beschreibung] #2
	\item[Erwartetes Ergebnis] #3
	\item[Tatsächliches Ergebnis] #4
	\item[Gefundene Fehler/Anmerkungen] #5
\end{description}}
\graphicspath{{images/}}

\begin{document}

%% titlepage.tex
%%

% coordinates for the bg shape on the titlepage
\newcommand{\diameter}{20}
\newcommand{\xone}{-15}
\newcommand{\xtwo}{160}
\newcommand{\yone}{15}
\newcommand{\ytwo}{-253}

\begin{titlepage}
% bg shape
\begin{tikzpicture}[overlay]
\draw[color=gray]
 		 (\xone mm, \yone mm)
  -- (\xtwo mm, \yone mm)
 arc (90:0:\diameter pt)
  -- (\xtwo mm + \diameter pt , \ytwo mm)
	-- (\xone mm + \diameter pt , \ytwo mm)
 arc (270:180:\diameter pt)
	-- (\xone mm, \yone mm);
\end{tikzpicture}
	\begin{textblock}{10}[0,0](4,2.5)
		\includegraphics[width=.3\textwidth]{images/kit_logo_de_4c_positiv.pdf}
	\end{textblock}
	\changefont{phv}{m}{n}	% helvetica	
	\vspace*{3.5cm}
	\begin{center}
		\fontsize{45}{50}\selectfont
  \textsc{Worthwhile} \\
  \textsc{Pflichtenheft}
		\vspace*{2cm}\\
		\LARGE
  Praxis der Softwareentwicklung -- WS 2011/2010 \\
  Automatisches Prüfen der Korrektheit von Programmen \\
  Projekt 4 -- Gruppe 2 \\
  \medskip
  \vspace*{2cm}
  \Large
  \begin{tabular}{|l|l|}
    \hline
    Leon Handreke & 123456 \\
    \hline
    Chris Hiatt & 123456 \\
    \hline
    Stefan Orf & 123456 \\
    \hline
    Joachim Priesner & 1579308 \\
    \hline
    Fabian Ruch & 123456 \\
    \hline
    Matthias Wagner & 1579342 \\
    \hline
  \end{tabular}
  \vspace*{2cm} \\
  \today \\
	Revision 0
	\end{center}
	\vspace*{1cm}

\begin{textblock}{10}[0,0](4,16.8)
\tiny{
	\iflanguage{english}
		{KIT -- University of the State of Baden-Wuerttemberg and National Laboratory of the Helmholtz Association}
		{KIT -- Universit\"at des Landes Baden-W\"urttemberg und nationales Forschungszentrum der Helmholtz-Gesellschaft}
}
\end{textblock}

\begin{textblock}{10}[0,0](14,16.75)
\large{
	\textbf{www.kit.edu}
}
\end{textblock}

\end{titlepage}


\tableofcontents
\clearpage

\section{Ablauf}

Bereits während der Implementierungsphase wurden viele Unit-Tests geschrieben, um während der Implementierung die Komponenten unabhängig vom Fortschritt der anderen Bereiche von Worthwhile testen zu können. Aus diesem Grund lag der initiale Fokus auf der Erarbeitung von Integrationstests und auf dem Schreiben von größeren Testprogrammen, um die einzelnen Komponenten mit einem möglichst breiten Spektrum an Eingabedaten testen zu können und die grenzen des Softwaresystems auszuloten. Nach der Beseitigung von offensichtlichen Fehlern wurden gegen Ende der 3 Wochen dauernden Validierungsphase noch Belastungs- und Testabdeckungstests durchgeführt.

\section{Verwendete Werkzeuge}
\subsection{Bugtracker}
Um die Kommunikation innerhalb des Teams zu erleichtern und stets problembezogen zu gestalten, wurde bereits während der Implementierungsphase  die Bugtracking-Software ``github Issues'' eingesetzt. Diese ist mit dem Quellcode-Repository integriert, sodass aus dem Bugtracker einfach auf Änderungen im Code verwiesen werden kann und sogar durch einen Eintrag in der Mitteilung zu einem Commit eine solche Referenz automatisch erzeugt werden kann.

Über den gesamten Entwicklungszeitraum hinweg wurden knapp über 100 Bugs angelegt. Davon bezogen sich etwa 30 auf die Benutzeroberfläche und jeweils ca. 20 auf die Beweiserschnittstelle und den Interpreter.

\subsection{Continous-Integration-Server}
Die Continous-Integration-Serversoftware Jenkins, die bereits während der Implementierungsphase verwendet wurde, um nach jeder Änderung die Kompilierbarkeit des Codes zu überprüfen, wurde auch in der Validierungsphase eingesetzt. Dabei wurden zusätzlich Plugins verwendet, die die Anzahl der Testfälle sowie die Anzahl der momentan erfolgreichen Tests grafisch anzeigen.

\begin{center}
	\begin{figure}[h] % place the figure here
		\includegraphics[width=13cm]{images/jenkins-test-trend.png}
		\caption{Entwicklung der Anzahl von Testfällen während der Validierungsphase}
	\end{figure}
\end{center}

\section{Überdeckungstests}
Zur Messung der Abdeckung des Codes durch die geschriebenen Testfälle wurde das Eclipse-Plugin EclEmma verwendet. Das Plugin hat gegenüber der Nutzung eines Programms auf der Kommandozeile den Vorteil, dass nicht durchlaufene Codezeilen in der Eclipse-Oberfläche markiert werden und so das Schreiben neuer Testfälle, die die Abdeckung erhöhen, vereinfacht wird.

Gemessen wurde die Abdeckung bei den Komponenten ``Beweiserschnittstelle'' und ``Interpreter''. Diese Module enthalten einen Großteil der automatisch testbaren Logik in Worthwhile und sind komplett von den Komponenten ``Benutzeroberfläche'' und ``Debugger'' entkoppelt, die durch manuell ausgeführte Testszenarien getestet wurden. Es wurde sowohl Anweisungs- als auch Zweigüberdeckung gemessen.

\subsection{Beweiserschnittstelle}
In der Komponente ``Beweiserschnittstelle'' wurde eine Anweisungsüberdeckung von 89,1\% und eine Zweigüberdeckung von 81,9\% erreicht. Dabei wurde in \texttt{WPStrategy}, der Komponente, die den weakest"=precondition-Kalkül implementiert, sogar eine Zweigabdeckung von 93,8\% erreicht. Durch die Verwendung des Visitor"=Entwurfsmusters wurden viele Fallunterscheidungen vermieden und damit das Testen in dieser Komponente stark vereinfacht.

\subsection{Interpreter}
In der Komponente ``Interpreter'' wurde eine Anweisungsüberdeckung von 76,7\% und eine Zweigüberdeckung von 66,9\% erreicht. Durch die vielfältige Auswahl von Testprogrammen wird dabei die Behandlung fast jedes Syntaxelementes mindestens einmal durchlaufen.

\section{Integrationstests}

Es wurden sowohl manuelle Integrationstests im Zuge der GUI"=Tests als
auch ständige automatische Integrationstests mit einem Archiv aus
Testprogrammen durchgeführt, wobei letztere zum einen den kompletten
Verifikationsprozess der Beweiserschnittstelle und zum anderen die
Interpreter"=Ausführung und "=Laufzeituntersuchungen durchlaufen.

Es wurden insgesamt 31 Testprogramme erstellt und teilweise auch mit einer Spezifikation versehen. Unter den Testprogrammen finden sich sowohl kleine Programme, die bestimmte Aspekte der Sprache und des Formelgenerators testen als auch größere Programme, die die Grenzen des Interpreters und des Formelgenerators testen sollen. So wurde der Sortierungsalgorithmus ``Bubblesort'', eine Heap-Datenstruktur sowie einige numerische Approximationsverfahren implementiert und teilweise auch die erwartete Funktionalität spezifiziert. Unter anderem bei ``Bubblesort'' wurden von Worthwhile teilweise Formeln generiert, die von Z3 nicht beweisbar, aber auch nicht widerlegbar waren.

\section{GUI-Tests}

Um die Funktionsfähigkeit der GUI sicherzustellen, wurden etliche manuelle Tests durchgeführt.

Die während der Validierungsphase erstellten Testfälle und deren Ergebnisse sind in Anhang~\ref{guitests-validierung} zusammengestellt.

Für die Ergebnisse der im Pflichtenheft festgelegten und im Zuge der Validierungsphase ausgeführten GUI-Tests siehe Tabelle~\ref{tests-pflichtenheft} in Anhang~\ref{guitests-pflichtenheft}.
\section{Profiling- und Performance-Tests}

Um einen Überblick über den Zeitaufwand der einzelnen Worthwhile-Komponenten zu gewinnen sowie zeitliche Hotspots zu identifizieren, wurden Profiling-Tests mit Unterstützung von VisualVM durchgeführt. Hierbei wurden einzelne, meist komplexere, Testprogramme aus der GUI gestartet und untersucht, um einen Eindruck von Worthwhile unter möglichst realen Bedingungen zu erlangen.

\subsection{Parser}

Das Parsen eines Programms ist abhängig von der Anzahl der vorkommenden Sprachkonstrukte. Beispielsweiße dauerte das Parsen des Programms bubblesort2.ww 111 ms. Eine eindeutige Aussage über die Dauer eines Parsevorgangs lässt sich schwer treffen, da die Anzahl sowie die Verschachtelungstiefe der Sprachkonstrukte eine große Rolle spielt. Jedoch benötigen die einzelne Sprachkonstrukte eine Bearbeitungszeit von Bruchteilen von Millisekunden bis hin zu wenigen Millisekunden. Im Testprogramm bubblesort2.ww stellte sich die Funktionsdeklaration mit 5,56 ms als am zeitaufwendigsten heraus.

\subsection{Interpreter}

Der Zeitbedarf für das Interpretieren ist genauso wie das Parsen abhängig von der Anzahl der Sprachkonstrukte. Jedoch muss hier nicht die Anzahl der vorkommenden sondern die der auszuführenden Elemente berücksichtigt werden. Deshalb lässt sich auch hier nicht eindeutig sagen wie viel Zeit ein Programm in der Ausführung benötigt, da insbesondere ein Prorgamm unter Umständen nie terminiert. Das Sortieren eines Arrays mit zehn Elementen mittels bubblesort2.ww benötigte etwa 18 Sekunden. Diese relativ hohe Zeit ist auf die ständig auftretenden Beweiseraufrufe zurück zu führen, welche annähernd 100 \% der Zeit ausmachten. Das Sortieren ohne die Beweiseraufrufe dauerte ca. 26 ms.

\subsection{Prover}

Das Überprüfen eines Testprogramms dauerte im Schnitt etwa 2 Sekunden. Am Beispiel der Testprogramme \texttt{test\_fibonacci.ww} und \texttt{bubblesort2.ww} lag die Beweiszeit bei 1865 ms bzw. 1895 ms. Da vor dem Beweisen ein Parsen des Codes und Typsystemüberprüfungen nötig sind, fallen etwa 100 bis 200 ms auf diese Aufgaben zurück. Die Transformation des Codes im wp-Kalkül benötigte für \texttt{bubblesort2.ww} 142 ms und für \texttt{test\_fibonacci.ww} 36 ms. Das Umwandeln einer einzelnen Annotation ins SMTLIB-Format betrug bei beiden Tests wenige Millisekunden (im Schnitt etwa 4 ms). Da jedoch die Annotationen einzeln umgewandelt und überprüft werden, liegt die Gesamtzeit der Umwandlungen ins SMTLIB-Format deutlich höher. Das Testprogramm \texttt{bubblesort2.ww} enthielt 13 und \texttt{test\_fibonacci.ww} 11 zu überprüfende Annotationen. Somit liegt der Zeitbedarf für das Umwandeln aller Annotation bei etwa 50 ms. Ebenso wird jede Annotation anschließend mittels externem Beweiser überprüft. Einzelne Annotationen zu beweisen, dauerte etwa 15 bis 17 ms und demnach lag der Gesamtzeitaufwand des Beweisers für das gesamte Programm bei 220 ms für \texttt{bubblesort2.ww} und bei 170 ms für \texttt{test\_fibonacci.ww}. Nach Überprüfung weiterer Testprogramme ergibt sich somit ein ungefährer Überblick über den anteiligen Zeitaufwand der Prover-Komponenten in Tabelle~\ref{zeitaufwandprover}.

\begin{table}[h]
\centering
\caption{Ungefährer Zeitaufwand der Prover-Komponenten}
\label{zeitaufwandprover}
\begin{tabular}{|l|l|}
\hline
\textbf{Komponente} & \textbf{Zeitaufwand} \\
\hline
Parser und Typsystem & 10\% \\
\hline
wp-Kalkül & 8\% \\
\hline
SMTLIB-Formatierung & 4\% \\
\hline
Beweiser (Z3) & 15\% \\
\hline
\end{tabular}
\end{table}

\section{Kompatibilitätstests}
Im Laufe der Entwicklung wurde auf Worthwhile auf die Kompatibilität mit verschiedenen Betriebsystemen, Rechnerarchitekturen und Java-Versionen getestet. Dabei wurde die Lauffähigkeit auf den Linux-Distributionen openSuse, Debian squeeze, Arch Linux, Fedora 14 sowie 16 und Gentoo sowie auf Windows 7 bestätigt. Bei einigen Tests wurde die offizielle Java-Implementation von Oracle verwendet, bei anderen die freie, auf OpenJDK basierence JVM ``IcedTea'' in Version 6 oder 7. Mit keiner der verwendeten virtuellen Maschinen wurden Probleme festgestellt.

Wider Erwarten war es nötig, für Systeme mit Windows und Linux sowie für amd64 und i386 Architekturen die Software mit unterschiedlichen Parametern zu bauen. Dies ist auf verschiedene platformspezifische Komponenten in der Eclipse-Platform zurückzuführen, die beispielsweise je nach Platform eine unterschiedliche Bibliothek zum Zeichnen der Benutzeroberfläche verwendet. Aller Worthwhile-spezifischer Code hat sich al platformunabhängig erwiesen.


\begin{appendix}
	\input{gui-tests-pflichtenheft}	
	\section{GUI-Testplan}
\label{guitests-validierung}

\subsection{Ausführen eines Programms (jeweils in den Modi Run, Debug, Prove)}
\guitest{Ausführen eines Programms mit Syntaxfehlern}{Aufruf des Ausführungsbefehls bei einem Programm, das einen beliebigen Syntaxfehler enthält.}{Fehlermeldung, Datei wird nicht ausgeführt}{OK}{--}
\guitest{Ausführen eines Programms mit Validierungsfehlern}{Aufruf des Ausführungsbefehls bei einem Programm, das einen beliebigen Validierungsfehler enthält.}{Fehlermeldung, Datei wird nicht ausgeführt}{OK}{--}
\guitest{Ausführen eines Programms aus dem Editor heraus}{Fokus auf Editorfenster setzen, Aufruf des Ausführungsbefehls}{Interpreter bzw. Specification Checker startet und führt die Datei aus, die im aktiven Editorfenster geöffnet ist.}{OK}{--}
\guitest{Ausführen eines Programms aus dem Projektexplorer heraus}{Fokus auf Projektexplorer setzen, Markieren einer Datei im Projektexplorer und Aufruf des Ausführungsbefehls}{Interpreter bzw. Specification Checker startet und führt die Datei aus, die im Projektexplorer markiert ist.}{(OK)}{\href{https://github.com/team-worthwhile/worthwhile/issues/85}{Bug \#85}: ``Prove it'' does not launch selection -- behoben}
\guitest{Ausführen einer Launch Configuration}{Erstellen einer neuen Launch Configuration unter \texttt{Run $\rightarrow$ Run configurations}, Setzen des Dateinamens auf eine existierende Datei und Klick auf "`Run"'}{Interpreter startet und führt die eingegebene Datei aus}{OK}{--}
\guitest{Ausführen einer Launch Configuration mit nicht existierender Datei}{Erstellen einer neuen Launch Configuration unter \texttt{Run $\rightarrow$ Run configurations}, Setzen des Dateinamens auf eine nicht existierende Datei oder den leeren String und Klick auf "`Run"'}{Fehlermeldung, Interpreter startet nicht}{(OK)}{\href{https://github.com/team-worthwhile/worthwhile/issues/86}{Bug \#86}: IllegalArgumentException when launching configuration with empty file name -- behoben}

\subsection{Setzen von Breakpoints in Editor und Debugger}

\guitest{Setzen und Löschen von Breakpoints}{Setzen und Entfernen von Breakpoints auf folgende Arten: \begin{enumerate}
	\item Doppelklick im Lineal des Editors
	\item Aufruf des Befehls \texttt{Toggle Breakpoint} im Kontextmenü des Lineals
	\item Aufruf des Befehls \texttt{Run $\rightarrow$ Toggle Breakpoint}
	\item Aufruf des Befehls \texttt{Run $\rightarrow$ Toggle Line Breakpoint}	
	\item (nur Entfernen) Rechtsklick auf einen Breakpoint in der Breakpoint-Übersicht und Aufruf von \texttt{Remove}
	\item (nur Entfernen) Rechtsklick auf einen Breakpoint in der Breakpoint-Übersicht und Aufruf von \texttt{Remove All}
\end{enumerate}}{Breakpoints werden korrekt eingefügt bzw. entfernt. Eingefügte Breakpoints werden im Lineal des Editors sowie in der Breakpoint-Übersicht angezeigt}{OK}{--}

\subsection{Eigenschaften von Breakpoints setzen}

\guitest{Aktivieren und Deaktivieren von Breakpoints}{Aktivieren und Deaktivieren eines Breakpoints in seinem Kontextmenü in der Breakpoint-Übersicht}{Breakpoint wird korrekt deaktiviert und im Debugger nicht beachtet, Breakpoint-Icon wird korrekt als gefüllter Kreis (aktiviert) bzw. leerer Kreis (deaktiviert) angezeigt.}{OK}{--}
\guitest{Bedingung setzen und über Neustart hinaus behalten}{Setzen einer Breakpoint-Bedingung über den entsprechenden Befehl im Kontextmenü des Breakpoints. Neustart von Worthwhile und erneutes Aufrufen des Befehls}{Breakpoint-Bedingung wird über den Neustart hinaus behalten und beim erneuten Aufrufen im Fenster für die Bedingungseingabe angezeigt}{OK}{--}

\subsection{Watchpoints}

\guitest{Watchpoints setzen und entfernen}{Watchpoints setzen und entfernen durch Platzieren des Cursors im Editorfenster und Auswahl von \texttt{Run $\rightarrow$ Toggle Watchpoint}}{Wenn sich der Cursor nicht über einer Variablendeklaration befindet: Fehlermeldung. Ansonsten: Setzen bzw, Entfernen eines Watchpoints mit Anzeige im Lineal des Editors und in der Breakpoint-Übersicht.}{(OK)}{\href{https://github.com/team-worthwhile/worthwhile/issues/87}{Bug \#87}: ``Toggle watchpoint'' causes line breakpoint to disappear and vice versa -- behoben}

\guitest{Aktivieren und Deaktivieren von Watchpoints}{Aktivieren und Deaktivieren eines Watchpoints in seinem Kontextmenü in der Breakpoint-Übersicht}{Watchpoint wird korrekt deaktiviert und im Debugger nicht beachtet, Watchpoint-Icon wird bei deaktiviertem Watchpoint ausgegraut angezeigt.}{OK}{--}

\subsection{Quick Fixes}

\guitest{Quickfix für "`Keine Newline am Ende der Datei"'}{Eingabe eines Programms, das am Ende kein Newline-Zeichen enthält}{Das Ende der Datei wird rot unterkringelt. Beim Überfahren mit der Maus wird das Einfügen eines Newline-Zeichens als \textit{Quick Fix} angeboten. Bei Auswahl des Quickfixes wird das Newline-Zeichen eingefügt und die Fehlermeldung verschwindet.}{OK}{--}
\guitest{Quickfix für "`Rückgabetyp der Funktion fehlt"'}{Eingabe eines Programms, das eine Funktionsdeklaration ohne Rückgabewert enthält, etwa \texttt{function test(Integer i) \{ \}}}{Die Funktionsdeklaration wird rot unterkringelt. Beim Überfahren mit der Maus wird das Einfügen eines Rückgabetyps als \textit{Quick Fix} angeboten. Bei Auswahl des Quickfixes wird der entsprechende Rückgabetyp eingefügt und die Fehlermeldung verschwindet.}{OK}{--}

\subsection{Auto Edit}

\guitest{\label{test-autoedit}Sämtliche AutoEdit-Ersetzungen ausprobieren}{Erstellen einer neuen Datei und Eingabe der auf S.~7 in Tabelle~2 des Entwurfsheftes dargestellten Alternativen für Unicode-Symbole}{Die Eingabe wird durch das entsprechende Unicode-Symbol gemäß der Tabelle ersetzt}{OK}{--}

\guitest{Eingabe von Kommentaren}{Bei aktiviertem AutoEdit-Feature wird ein Mehrzeilenkommentar sowie ein Einzeilenkommentar eingegeben.}{Die Kommentare lassen sich korrekt eingeben, insbesondere kollidiert die Eingabe von \texttt{/* */} nicht mit der Ersetzung von \texttt{*} durch \texttt{$\cdot$}.}{OK}{--}

\guitest{Keine Ersetzungen in Kommentaren}{Es werden ein Einzeilenkommentar und ein Mehrzeilenkommentar eingegeben. In beiden Kommentaren werden die unter \ref{test-autoedit} definierten Eingaben ausgeführt.}{Es findet keine Ersetzung durch Unicode-Symbole statt.}{OK}{--}

\subsection{Perspektive}

\guitest{Schließen der Worthwhile-Perspektive}{Die Worthwhile-Perspektive wird über das Kontextmenü in der Perspektiven-Werkzeugleiste geschlossen.}{Die Perspektive wird ohne Fehlermeldung geschlossen}{OK}{--}
\guitest{Öffnen der Worthwhile-Perspektive}{Über den Befehl \texttt{Window $\rightarrow$ Open Perspective} wird die Worthwhile-Perspektive geöffnet.}{Die Perspektive öffnet sich mit der Standardanordnung der Fenster.}{OK}{--}

\subsection{Einstellungen}

\guitest{Öffnen des Dateiauswahldialogs für den Beweiser (1)}{Eingabe eines gültigen Dateinamens im Feld zur Eingabe des Pfades zum Beweiser und Klick auf \texttt{Browse}}{Dateiauswahldialog öffnet sich und wählt die eingegebene Datei als Standard aus.}{OK}{--}

\guitest{Öffnen des Dateiauswahldialogs für den Beweiser (2)}{Eingabe eines ungültigen oder nicht existierenden Dateinamens im Feld zur Eingabe des Pfades zum Beweiser und Klick auf \texttt{Browse}}{Dateiauswahldialog öffnet sich ohne Fehlermeldung und zeigt einen beliebigen (Standard-)Ort an.}{OK}{--}

\guitest{Wiederherstellen der Standardwerte}{Klick auf \texttt{Restore Defaults} im Einstellungsfenster}{Einstellungen werden auf die Standardwerte zurückgesetzt.}{OK}{--}

\subsection{Formatierer}

\guitest{Formatieren eines Dokuments}{Öffnen eines Dokuments, das alle Syntaxelemente enthält und Aufruf des Formatierers über den entsprechenden Befehl}{Die Datei wird formatiert, wobei die syntaktische Korrektheit erhalten bleibt. Mehrfaches Formatieren ändert die Datei nicht mehr.}{(OK)}{\href{https://github.com/team-worthwhile/worthwhile/issues/88}{Bug \#88}: Formatter breaks syntactic correctness and modifies the program iteratively -- behoben}

\subsection{Dateiverwaltung}

\guitest{Anlegen und Löschen einer Worthwhile-Datei}{Anlegen einer neuen Worthwhile-Datei mithilfe des Dateiassistenten. Löschen der Datei}{Die Datei wird korrekt angelegt und nach einer Nachfrage wieder gelöscht.}{OK}{--}
\guitest{Anlegen und Löschen eines Worthwhile-Projekts}{Anlegen eines neuen Worthwhile-Projekts mithilfe des Projektassistenten. Löschen des Projekts}{Das Projekt wird korrekt angelegt und nach einer Nachfrage wieder gelöscht.}{OK}{--}

\subsection{Dynamische Hilfe}

\guitest{Dynamische Hilfe im Editor}{Öffnen der dynamischen Hilfe mittels des Befehls \texttt{Help $\rightarrow$ Dynamic Help}. Öffnen eines Programms, das alle Syntaxelemente enthält. Platzieren des Cursors auf einem Schlüsselwort}{Das zum Schlüsselwort gehörige Hilfethema wird im Fenster der dynamischen Hilfe angezeigt.}{OK}{--}

\subsection{Outline}

\guitest{Outline}{Öffnen des Outline-Fensters und Öffnen eines Programms in Editor}{Im Outline-Baum werden alle Axiome, Funktionsdeklarationen und der Main-Block angezeigt. Bei Funktionsdeklarationen werden darüber hinaus Vor- und Nachbedingungen sowie der Funktionsrumpf angezeigt.}{OK}{--}

\subsection{Debugger: Breakpoints}

\guitest{Bedingte Breakpoints}{Debuggen eines Programms, das folgende bedingten Breakpoints enthält: \begin{enumerate}
	\item Mit syntaktisch korrekter Bedingung
	\item Mit Syntaxfehler in Bedingung
	\item Mit Semantikfehler in Bedingung
	\item Mit Funktionsaufruf in Bedingung
\end{enumerate}}{Korrektes Anhalten, wenn die Bedingung erfüllt ist. Fehlermeldung bei Erreichen des Breakpoints, wenn ein Syntax- oder Semantikfehler vorliegt. Korrekte Evaluation des Funktionsaufrufes, wobei in der Funktion gesetzte Breakpoints ignoriert werden.}{(OK)}{\href{https://github.com/team-worthwhile/worthwhile/issues/89}{Bug \#89}: UnsupportedOperationException on breakpoint with condition – behoben}

\guitest{Breakpoints setzen, entfernen, deaktivieren und aktivieren während der  Ausführung}{Debuggen eines Programms, dabei Setzen und Entfernen, Aktivieren und Deaktivieren von Breakpoints und Watchpoints während der Ausführung des Programms}{An deaktivierten und entfernten Breakpoints/Watchpoints wird nicht angehalten. An neu gesetzten und aktivierten Breakpoints/Watchpoints wird angehalten}{OK}{--}

\subsection{Debugger: Variableninspektion}

\guitest{Anzeige des Wertes skalarer Variablen}{Debuggen und Pausieren eines  Programms, dann Aufruf der Variablenliste}{Die Werte skalarer Variablen (Integer, Boolean) werden gemäß ihrem Typ angezeigt}{OK}{--}
\guitest{Ändern des Wertes skalarer Variablen}{Debuggen und Pausieren eines  Programms, dann Aufruf der Variablenliste. Eingabe eines neuen (gültigen) Wertes für eine Variable}{Der Variableninhalt wird geändert, der Interpreter übernimmt bei der weiteren Ausführung den neuen Wert der Variablen}{OK}{--}
\guitest{Anzeige des Wertes von Arrays}{Debuggen und Pausieren eines  Programms, dann Aufruf der Variablenliste}{Die Werte von Arrays werden gemäß ihrem Typ angezeigt. Durch Aufklappen der Liste können die einzelnen Werte betrachtet werden.}{OK}{--}
\guitest{Ändern des Wertes von Arrays}{Debuggen und Pausieren eines  Programms, dann Aufruf der Variablenliste. Eingabe eines neuen (gültigen) Wertes für eine Arrayvariable}{Der Variableninhalt wird geändert, der Interpreter übernimmt bei der weiteren Ausführung den neuen Wert der Variablen}{OK}{--}
\guitest{Werteänderung mit Funktionsaufruf}{Debuggen und Pausieren eines  Programms, dann Aufruf der Variablenliste. Eingabe eines Funktionsaufrufs als neuen Wert}{Der Wert der Variablen wird auf den Rückgabewert der aufgerufenen Funktion gesetzt}{OK}{--}
\guitest{Werteänderung mit ungültigem Wertetyp}{Debuggen und Pausieren eines  Programms, dann Aufruf der Variablenliste. Eingabe eines neuen Wertes, der nicht dem Wertetyp der Variablen entspricht, oder Eingabe einer leeren Zeichenkette als neuen Wert}{Fehlermeldung, der Variableninhalt wird nicht geändert}{(OK)}{\href{https://github.com/team-worthwhile/worthwhile/issues/90}{Bug \#90}: Debugger allows setting variable value to wrong type -- behoben}

\subsection{Debugger: Ausdrucksauswertung}

\guitest{Auswertung von Ausdrücken}{Debuggen und Pausieren eines Programms, dann Aufruf des Ausdrucks-Auswertungsfensters. Eingabe folgender Ausdrücke: \begin{enumerate}
	\item Einfacher Ausdruck, der keine Variable enthält (z.B.~\texttt{1+1})
	\item Einfacher Ausdruck, der eine Variable enthält (z.B.~\texttt{i + 2})
	\item Ausdruck mit Syntaxfehler
	\item Ausdruck mit Semantikfehler
	\item \texttt{1/0} als Ausdruck
	\item Ausdruck mit Array als Rückgabetyp
\end{enumerate}}{Korrekte Auswertung der gültigen Ausdrücke, Anzeige einer Fehlermeldungen bei ungültigen oder nicht auswertbaren Ausdrücken}{(OK)}{\href{https://github.com/team-worthwhile/worthwhile/issues/91}{Bug \#91}: Watch expressions are not validated –  behoben \\ href{https://github.com/team-worthwhile/worthwhile/issues/113}{Bug \#113}: EmptyStackException when evaluating 1/0 as expression}

\subsection{Debugger-Funktionen}

\guitest{Step-Funktionen}{Debuggen und Pausieren eines Programms, dann jeweils Ausführung der Funktionen \texttt{Step into} und \texttt{Step over}, wenn der Debugger auf folgenden Anweisungen steht: \begin{enumerate}
	\item Normale Anweisung ohne Funktionsaufruf
	\item Normale Anweisung mit Funktionsaufruf
	\item Kopf einer Schleife
	\item Bedingte Verzweigung (if)
\end{enumerate}}{Bei \texttt{Step over} wird die komplette nächste Anweisung übersprungen; in Schleifen, Bedingungen und Funktionsaufrufe wird nicht gesprungen. Bei \texttt{Step into} wird in Schleifen, Bedingungen und Funktionsaufrufe gesprungen.}{OK}{--}

\guitest{Terminierung}{Debuggen eines Programms und Aufruf des Befehls \texttt{Terminate}}{Die Ausführung des Programms wird abgebrochen.}{OK}{--}

\guitest{Anhalten}{Debuggen eines Programms und Aufruf des Befehls \texttt{Suspend}}{Die Ausführung des Programms wird angehalten; die aktuelle Anweisungszeile wird hervorgehoben.}{OK}{--}

\guitest{Weiter ausführen}{Debuggen und Pausieren eines Programms, dann Aufruf des Befehls \texttt{Resume}}{Die Ausführung des Programms wird fortgesetzt und erst beim nächsten Breakpoint oder bei der nächsten fehlgeschlagenen Annotation wieder angehalten.}{OK}{--}

\subsection{Interpreter}

\guitest{Auswertung von Annotationen}{Debuggen eines Programms, das folgende Annotationen enthält: \begin{enumerate}
	\item Eine gültige Annotation
	\item Eine ungültige Annotation
	\item Eine Annotation mit unbekanntem Gültigkeitswert
\end{enumerate}}{Die gültige Annotation wird grün markiert, die beiden anderen rot. Der Interpreter hält außerdem bei den rot markierten Annotationen an.}{OK}{--}
\guitest{Fehler bei Aufruf des Beweisers}{Einstellung eines ungültigen Beweiserpfades in den Einstellungen und Debuggen eines Programms, das eine Annotation enthält.}{Annotation wird rot markiert, Interpreter stoppt und gibt eine Fehlermeldung aus.}{(OK)}{\href{https://github.com/team-worthwhile/worthwhile/issues/58}{Bug \#58}: Interpreter throws ugly error message when prover is missing}
\guitest{Division durch Null}{Es wird ein Programm ausgeführt, das den Ausdruck \texttt{1/0} oder \texttt{1\%0} enthält.}{Fehlermeldung und Programmabbruch}{OK}{--}
\end{appendix}

\clearpage
\appendix

\end{document}
